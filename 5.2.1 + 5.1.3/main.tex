\documentclass[12pt]{article}
\usepackage[top=1cm, bottom=2cm, right=1cm, left=1cm]{geometry}
\usepackage{amsfonts, amssymb, amsmath, hyperref}
\usepackage{graphicx}
\usepackage[T1, T2A]{fontenc}% T2A for Cyrillic font encoding
\usepackage[english, russian]{babel}
\usepackage[justification=centering]{caption}
\usepackage{wrapfig}
\usepackage{placeins}
\usepackage{subcaption}
\usepackage{multirow}
\usepackage{indentfirst}
\usepackage{needspace}
\usepackage{float}
\floatplacement{table}{H}

\newcommand{\Equip}[3]{
	
	{\bf #1:} $\Delta = \pm #2\; #3$}
\newcommand{\equip}[1]{
	
	{\bf #1}}

\begin{document}
\title{\textbf{Лабораторная работа 5.2.1 + 5.1.3}\\ [2pt]{Опыт Франка-Герца и изучение рассеяния медленных электронов на атомах (эффект Рамзауэра).}}
\date{\today}
\author{Павлов Матвей}

\maketitle

\section*{Аннотация}
Методом электронного возбуждения измерить энергию первого уровня атома гелия в динамическом и статическом состояниях. Исследовать энергетическую зависимость вероятности рассеяния электронов атомами ксенона, определить уровни электронов, при которых наблюдается "просветление" ксенона, и оценить размер его внешней электронной оболочки.

\section*{Теоретические сведения}

\subsection*{Опыт Франка-Герца}

Разреженный одноатомный газ (гелий) заполняет трёхэлектродную лампу. Электроны, испускаемые катодом, ускоряются в электрическом поле между катодом и сетчатым анодом. При движении к аноду электроны сталкиваются с атомами гелия. Если энергия электрона недостаточна для возбуждения атома, происходят упругие соударения, при которых электроны почти не теряют энергии из-за малой массы.

При увеличении разности потенциалов энергия электронов становится достаточной для возбуждения атомов. При неупругих столкновениях кинетическая энергия электрона передаётся атомному электрону, вызывая переход на другой энергетический уровень или ионизацию.

С ростом анодного потенциала ток в лампе сначала увеличивается. Когда энергия электронов достигает порога возбуждения атомов, ток коллектора резко уменьшается, так как электроны теряют энергию при неупругих соударениях и не могут преодолеть задерживающий потенциал. При дальнейшем увеличении потенциала анода ток снова возрастает: электроны после неупругих соударений успевают набрать достаточную энергию для преодоления задерживающего потенциала.

\begin{figure}[H]
	\centering
	\includegraphics[width=0.25\textwidth]{images/Screenshot_1}
	\caption{Характер зависимости $I (U)$}
	\label{fig:screenshot1}
\end{figure}


Кинетическая энергия электрона 1 уровня равна:
\begin{equation}\label{key}
	E = \overline{e} \Delta V \left[\text{эВ}\right],
\end{equation}
где $ \Delta V $ -- разность между двумя пиками.

\subsection*{Эффект Рамзауэра}

\begin{wrapfigure}{}{0.3\textwidth}
	\includegraphics[width=1.0\linewidth]{images/1}
	\caption{Качественная картина результатов измерения упругого рассеяния электронов в аргоне}
	\label{fig:1}
\end{wrapfigure}
Эффективное сечение реакции -- это величина, характеризующая вероятность перехода системы двух сталкивающихся частиц в результате их рассеяния (упругого или неупругого) в определенное конечное состояние. Сечение $ \sigma $ равно отношению числа $ N $ таких переходов в единицу времени к плотности потока рассеиваемых частиц $ n v $, падающих на мишень, т. е. к числу частиц, проходящих в единицу времени через единичную площадку, перпендикулярную к их скорости $ v $ ($ n $ -- плотность числа падающих частиц).
\begin{equation}\label{eq:sigma}
	\sigma = \frac{N}{n v}.
\end{equation}
Таким образом, сечение имеет размерность площади.

Качественно результат экспериментов Рамзауэра при энергии электронов порядка десятков эВ показан на рис. \ref{fig:1}.
По мере уменьшения энергии электрона от нескольких десятков электрон-вольт поперечное сечение его упругого рассеяния растет. Однако при энергиях меньше 16 эВ в случае аргона сечение начинает уменьшаться, а при $ E \sim 1 $ эВ практически равно нулю, т. е. аргон становится прозрачным для электронов. При дальнейшем уменьшении энергии электронов сечение рассеяния опять начинает возрастать. Это поведение поперечного сечения свойственно не только атомам аргона, но и атомам всех инертных газов. Такое поведение электронов нельзя объяснить с позиций классической физики. Объяснение этого эффекта потребовало учета волновой природы электронов. Схема эксперимента Рамзауэра показана, на рис. \ref{fig:2}.

\begin{figure}[h!]
	\centering
	\includegraphics[width=0.5\linewidth]{images/2}
	\caption{Схема установки для измерения сечения рассеяния электронов в газах}
	\label{fig:2}
\end{figure}

С точки зрения квантовой теории, внутри атома потенциальная энергия налетающего электрона $ U $ отлична от нуля, скорость электрона изменяется, становясь равной $ v' $ в соответствии с законом сохранения энергии
\begin{equation*}
	E = \frac{m v^2}{2} = \frac{m v'^2}{2}+ U,
\end{equation*}
а значит, изменяется и длина его волны де Бройля. Таким образом, по отношению к электронной волне атом ведет себя как преломляющая среда с относительным показателем преломления
\begin{equation*}
	n = \frac{\lambda}{\lambda'} = \sqrt{1-\frac{U}{E}}.
\end{equation*}

Коэффициент прохождения электронов максимален при условии
\begin{equation}\label{eq:at}
	\sqrt{\frac{2 m (E+U_0)}{\hbar^2}}l = \pi n;\; n \in \mathbb{N}_1,
\end{equation}
где $ U_0 $ -- глубина потенциальной ямы.

Это условие легко получить, рассматривая интерференцию электронных волн де Бройля в атоме. Движущемуся электрону соответствует волна де Бройля, длина которой определяется соотношением $ \lambda = h/m v $. Если кинетическая энергия электрона невелика, то $ E = m v^2/2 $ и $ \lambda = h/\sqrt{2 m E} $. При движении электрона через атом длина волны де Бройля становится меньше и равна $ \lambda' = h/\sqrt{2 m (E+U_0)} $ где $ U_0 $ — глубина атомного потенциала. При этом, волна де Бройля отражается от границ атомного потенциала, т. е. от поверхности атома, и происходит интерференция прошедшей через атом волны 1 и волны 2, отраженной от передней и задней границы атома (эти волны когерентны). Прошедшая волна 1 усилится волной 2, если геометрическая разность хода между ними $ \Delta = 2 l = \lambda' $, что соответствует условию первого интерференционного максимума, т. е. при условии
\begin{equation}\label{eq:condition}
	2 l = \frac{h}{\sqrt{2 m (E_1 + U_0)}}
\end{equation}
Прошедшая волна ослабится при условии
\begin{equation}\label{eq:condition2}
	2 l = \frac{3}{2}\frac{h}{\sqrt{2 m (E_1 + U_0)}}
\end{equation}

Из \eqref{eq:condition} и \eqref{eq:condition2}, можно получить
\begin{equation}\label{eq:radius}
	l = \frac{h \sqrt{5}}{\sqrt{32 m (E_2-E_1)}}.
\end{equation}
Оттуда же можно найти эффективную глубину потенциальной ямы атома:
\begin{equation}\label{eq:atomPit}
	U_0 = \frac{4}{5}E_2-\frac{9}{5} E_1.
\end{equation}


Уравнение вольт-амперной характеристики тиратрона:
\begin{equation}\label{eq:VAH}
	I_\text{а} = I_0 \exp (-C \omega (V));\; C = L n_\text{а} \Delta_\text{а},
\end{equation}
где $ I_0 = e N_0 $ -- ток катода, а $ I_\text{а} = e N_\text{а} $ -- ток анода.
Отсюда определяется вероятность рассеяния электрона в зависимости от его энергии:
\begin{equation}\label{eq:probable}
	\omega (V) = -\frac{1}{C} \ln \frac{I_\text{а}(V)}{I_0}.
\end{equation}


\section*{Экспериментальная установка}

\subsection*{Опыт Франка-Герца}

Схема экспериментальной установки отображена на рис. \ref{fig:screenshot2} и \ref{fig:screenshot3}.

\begin{figure}[h!]
	\centering
	\includegraphics[height=0.4\textwidth]{images/Screenshot_2}
	\caption{Принципиальная схема установки}
	\label{fig:screenshot2}
\end{figure}

\begin{figure}[h!]
	\centering
	\includegraphics[width=0.4\textwidth]{images/Screenshot_3}
	\caption{Блок-схема экспериментальной установки}
	\label{fig:screenshot3}
\end{figure}


В работе используются:

\begin{itemize}
	\item \Equip{Вольтметры}{0.1}{В}
	\item \Equip{Осциллограф}{0.4}{В}
	\item \Equip{Миллиамперметр}{0.5}{мА}
	\item \equip{Блок источников питания}
	\item \equip{Газонаполненная лампа} (гелий)
\end{itemize}

\subsection*{Эффект Рамзауэра}

Схема экспериментальной установки отображена на рис. \ref{fig:3}.
\begin{figure}[h!]
	\centering
	\includegraphics[width=0.6\textwidth]{images/3}
	\caption{Схема экспериментальной установки}
	\label{fig:3}
\end{figure}
В данной работе для изучения эффекта Рамзауэра используется тиратрон ТГЗ-01/1.3Б, заполненный инертным газом. Электроны, эмитируемые катодом тиратрона, ускоряются напряжением $ V $, 
приложенным между катодом и ближайшей к нему сеткой. Затем электроны рассеиваются на атомах инертного газа (ксенона). Все сетки соединены между собой и имеют одинаковый потенциал,
примерно равный потенциалу анода. Поэтому между первой сеткой и анодом практически нет поля. Рассеянные электроны отклоняются в сторону и уходят на сетку, а оставшаяся часть электронов 
достигает анода и создаёт анодный ток $I_\text{а}$. Таким образом, поток электронов $ N(x) $ (т. е. число электронов, проходящих через поперечное сечение лампы в точке $ x $ в единицу времени) 
уменьшается с ростом $ x $ от начального значения $X$ y катода (в точке $x = 0$) до некоторого значения $ N_\text{а} $ у анода (в точке
$ x=L $).


В работе используются:
\begin{itemize}
	\item \Equip{Вольтметры}{0.01}{В}
	\item \Equip{Осциллограф}{0.2}{В} по оси X
	\item \equip{Блок источников питания}
	\item \equip{Тиратрон ТГ3}
\end{itemize}


\section*{Ход работы}

\subsection*{Опыт Франка-Герца}

\subsubsection*{Динамический режим}

При максимальном ускоряющем напряжении измерим на экране расстояние между максимумами и между минимумами осциллограммы. Измерения проведём при трёх значениях задерживающего напряжения: 2, 4 и 6 В. Результаты измерений занесём в таблицу 1. Фотографии полученных осциллограмм \ref{fig:osc}.

\begin{table}[H]
	\centering
	\begin{tabular}{|c|c|c|c|c|c|c|} \hline
		$U_\text{з}, \;В$ & $U_{\text{min}_1}$, В & $U_{\text{max}_1}$, В & $U_{\text{min}_2}$, В & $U_{\text{max}_2}$, В & $\Delta U, \; \text{В}$ & $E,\; \text{эВ}$ \\ \hline
		2          & $22 \pm 2$ & $17.5 \pm 2$ & $38.5 \pm 2$ & $32.5 \pm 2$ & $15\pm 4$ & $15\pm 4$        \\ \hline
		4          & $23 \pm 2$ & $18 \pm 2 $ & $41 \pm 2 $ & $ 34 \pm 2 $ & $16\pm 4$ & $16\pm 4$        \\ \hline
		6          & $24.5 \pm 2$ & $19 \pm 2$ & $42.5 \pm 2 $ & $35 \pm 2$ & $16\pm 4$ & $16\pm 4$        \\ \hline
	\end{tabular}
	\label{tab:1}
	\caption{Результаты динамического измерения}
\end{table}

После усреднения получаем:

\[
E = 15.7 \pm 3.5 \text{ эВ}
\]

Погрешность считаем так:
\[
\sigma_{E} = E \cdot \sqrt{\left(\frac{\sigma_{E1}}{E1}\right)^2 + \left(\frac{\sigma_{E2}}{E2}\right)^2 + \left(\frac{\sigma_{E3}}{E3}\right)^2}
\]

\begin{figure}[h!]
	\centering
	\includegraphics[width=0.29\textwidth]{images/osc1.png}
	\includegraphics[width=0.29\textwidth]{images/osc2.png}
	\includegraphics[width=0.29\textwidth]{images/osc3.png}
	\caption{Осицллограммы для $V_\text{з} = 2, 4, 6 \text{ В}$}
	\label{fig:osc}
\end{figure}

При этом табличное значение данной величины составляет
\begin{center}
    $E_{th} = 21.6$ эВ
\end{center}
С учётом погрешности, экспериментальные данные близки к теоретическим.

\subsubsection*{Статический режим}

Снимем зависимость коллекторного тока от анодного напряжения $I_\text{k} = f(V_\text{a})$ для значений задерживающего напряжения 2, 4 и 6 В. Результаты измерений занесём в таблицы 2-4.

\[
\sigma_U = 0.1 \text{ В}
\]

\[
\sigma_\text{I} = 1 \text{ мкА}
\]

\begin{table}[H]
    \centering
    \begin{tabular}{|c|c|c|c|c|c|c|c|c|c|c|c|c|c|c|} \hline
        $U$, В & 78.08 & 65.18 & 60.65 & 54.97 & 50.82 & 44.56 & 40.23 & 35.48 & 30.56 & 25.09 & 20.87 & 14.97 & 10.43 & 4.73 \\ \hline
        $I$, мкА & 375 & 328 & 318 & 293 & 263 & 241 & 243 & 251 & 203 & 142 & 154 & 135 & 102 & 56 \\ \hline
        $U$, В & 22.07 & 23.99 & 26.66 & 28.17 & 31.30 & 29.92 & 32.46 & 33.68 & 34.59 & 36.75 & 37.98 & 38.76 & 40.97 & 41.58 \\ \hline
        $I$, мкА & 126 & 130 & 161 & 179 & 215 & 196 & 227 & 240 & 247 & 255 & 254 & 252 & 236 & 234 \\ \hline
        $U$, В & 42.72 & 43.45 & 44.08 & 45.54 & 46.76 & 48.58 & 16.88 & 17.69 & 18.39 & 19.12 & 19.48 & 20.27 & 21.12 & \\ \hline
        $I$, мкА & 233 & 236 & 238 & 244 & 248 & 259 & 155 & 162 & 166 & 168 & 168 & 165 & 151 & \\ \hline
    \end{tabular}
    \label{tab:2}
    \caption{ВАХ при $V_\text{з} = 2$ В}
\end{table}

\begin{table}[H]
    \centering
    \begin{tabular}{|c|c|c|c|c|c|c|c|c|c|c|c|c|c|c|} \hline
        $U$, В & 78.17 & 63.25 & 60.09 & 55.00 & 50.09 & 45.05 & 40.15 & 35.23 & 30.56 & 25.05 & 20.10 & 15.03 & 10.01 & 5.30 \\ \hline
        $I$, мкА & 278 & 255 & 252 & 223 & 190 & 175 & 203 & 213 & 157 & 75 & 157 & 129 & 88 & 47 \\ \hline
        $U$, В & 17.05 & 18.14 & 19.10 & 20.99 & 22.04 & 23.06 & 22.50 & 24.14 & 23.56 & 26.06 & 27.04 & 28.03 & 29.16 & 33.23 \\ \hline
        $I$, мкА & 146 & 153 & 158 & 152 & 141 & 79 & 126 & 70 & 71 & 89 & 107 & 120 & 138 & 193 \\ \hline
        $U$, В & 34.09 & 36.87 & 38.20 & 41.23 & 42.30 & 43.03 & 44.07 & 46.26 & 47.47 & 48.82 & 51.71 & 52.18 & & \\ \hline
        $I$, мкА & 202 & 221 & 218 & 194 & 186 & 183 & 178 & 177 & 179 & 183 & 202 & 201 & & \\ \hline
    \end{tabular}
    \label{tab:3}
    \caption{ВАХ при $V_\text{з} = 4$ В}
\end{table}

\begin{table}[H]
    \centering
    \begin{tabular}{|c|c|c|c|c|c|c|c|c|c|c|c|c|c|c|} \hline
        $U$, В & 78.09 & 63.61 & 60.37 & 55.01 & 50.20 & 45.11 & 40.10 & 35.13 & 30.10 & 25.01 & 20.10 & 15.03 & 10.18 & 5.00 \\ \hline
        $I$, мкА & 161 & 163 & 162 & 138 & 115 & 120 & 153 & 157 & 89 & 34 & 141 & 110 & 70 & 27 \\ \hline
        $U$, В & 15.97 & 17.23 & 18.31 & 19.16 & 21.06 & 22.04 & 23.17 & 24.02 & 23.57 & 23.45 & 26.06 & 27.01 & 28.35 & 29.15 \\ \hline
        $I$, мкА & 116 & 126 & 134 & 140 & 140 & 133 & 106 & 39 & 50 & 58 & 38 & 45 & 62 & 75 \\ \hline
        $U$, В & 31.03 & 33.50 & 36.34 & 38.2 & 42.32 & 44.05 & 46.96 & 48.39 & 51.23 & 52.81 & 53.98 & 56.07 & 57.69 & 47.61 \\ \hline
        $I$, мкА & 104 & 138 & 165 & 167 & 139 & 128 & 114 & 112 & 120 & 127 & 133 & 147 & 155 & 113 \\ \hline
        $U$, В & 45.96 & 43.41 & & & & & & & & & & & & \\ \hline
        $I$, мкА & 119 & 133 & & & & & & & & & & & & \\ \hline
    \end{tabular}
    \label{tab:4}
    \caption{ВАХ при $V_\text{з} = 6$ В}
\end{table}

Представим графики вольт-амперных характеристик трёхэлектродной лампы при разных задерживающих напряжениях на рис. \ref{fig:vac} и определим энергию возбуждения первого уровня атома гелия.

\begin{figure}[h]
    \centering
    \includegraphics[width=\textwidth]{images/graph_1.png}
    \caption{Вольт-амперные характеристики трёхэлектродной вакуумной лампы при разных значениях запирающего напряжения}
    \label{fig:vac}
\end{figure}

\begin{table}[H]
	\centering
	\begin{tabular}{|c|c|c|c|c|c|c|} \hline
		$U_\text{з}, \;В$ & $U_{\text{min}_1}$, В & $U_{\text{max}_1}$, В & $U_{\text{min}_2}$, В & $U_{\text{max}_2}$, В & $\Delta U, \; \text{В}$ & $E,\; \text{эВ}$ \\ \hline
		2          & $22.07 \pm 0.10$ & $19.12 \pm 0.10$ & $42.72 \pm 0.10$ & $36.75\pm 0.10$ & $17.63\pm 0.20$ & $17.63\pm 0.20$        \\ \hline
		4          & $24.14 \pm 0.10$ & $19.10 \pm 0.10$ & $45.05 \pm 0.10$ & $36.87 \pm 0.10$ & $17.77\pm 0.20$ & $17.77\pm 0.20$        \\ \hline
		6          & $25.01 \pm 0.10$ & $20.1 \pm 0.10$ & $48.39 \pm 0.10$ & $38.20 \pm 0.10$ & $18.10\pm 0.20$ & $18.10\pm 0.20$        \\ \hline
	\end{tabular}
	\label{tab:1}
	\caption{Результаты статического измерения}
\end{table}

После усреднения получаем:

\[
E = 17.83 \pm 0.35 \text{ эВ}
\]

При этом табличное значение данной величины составляет
\begin{center}
    $E_{th} = 21.6$ эВ
\end{center}
С учётом погрешности, экспериментальные данные получаются одного порядка с теоретическими, но не совпадают в пределах погрешности.

\subsection*{Эффект Рамзауэра}

\subsubsection*{Динамический режим}

\begin{table}[H]
	\centering
	\begin{tabular}{|c|c|c|c|c|c|c|} \hline
		$U_\text{накала}, \;В$ & $U_{\text{min}_1}$, В & $U_{\text{max}_1}$, В & $U_{\text{min}_2}$, В & $U_{\text{max}_2}$, В & $\Delta U, \; \text{В}$ & $E,\; \text{эВ}$ \\ \hline
		2.946          & $ \pm $ & $ \pm $ & $ \pm $ & $ \pm $ & $\pm $ & $ \pm $        \\ \hline
		2.598          & $ \pm $ & $ \pm $ & $ \pm $ & $ \pm $ & $\pm $ & $ \pm $        \\ \hline
	\end{tabular}
	\label{tab:1}
	\caption{Результаты статического измерения}
\end{table}

\subsubsection*{Статический режим}

\begin{table}[H]
	\centering
	\begin{tabular}{|c|c|c|c|c|c|c|c|c|c|c|c|c|c|c|} \hline
		$U$, В & 8,367 & 7,632 & 7,220 & 7,022 & 6,866 & 6,629 & 8,584 & 8,721 & 9,005 & 9,271 & 9,403 & 9,652 & 9,870 & 10,028 \\ \hline
		$I$, мкА & 107,82 & 92,20 & 87,19 & 84,99 & 83,29 & 81,45 & 111,09 & 114,45 & 122,49 & 128,61 & 131,68 & 142,01 & 155,01 & 165,60 \\ \hline
		$U$, В & 10,283 & 10,808 & 8,486 & 6,467 & 6,268 & 6,003 & 5,780 & 5,611 & 5,398 & 5,180 & 5,016 & 4,832 & 4,624 & 4,422 \\ \hline
		$I$, мкА & 178,03 & 196,34 & 111,74 & 83,07 & 81,98 & 80,98 & 80,53 & 80,44 & 80,69 & 81,25 & 82,0 & 83,09 & 84,78 & 86,84 \\ \hline
		$U$, В & 4,222 & 4,075 & 3,753 & 3,498 & 3,248 & 3,013 & 2,812 & 2,604 & 2,447 & 2,270 & 2,072 & 1,842 & 1,596 & 1,366 \\ \hline
		$I$, мкА & 89,37 & 91,56 & 97,47 & 103,47 & 110.67 & 119,02 & 127,70 & 138,95 & 148,98 & 161,94 & 178,09 & 194,76 & 198,01 & 171,23 \\ \hline
		$U$, В & 0,902 & 0,517 & 0,783 & 0,921 & 1,062 & 1,109 & & & & & & & & \\ \hline
		$I$, мкА & 44,07 & 1,59 & 20,21 & 50,30 & 91,30 & 105,81 & & & & & & & & \\ \hline
	\end{tabular}
	\caption{Вольт-амперная характеристика}
	\label{tab:4}
\end{table}

\section*{Вывод}

Таким образом, в ходе работы были получены следующие результаты:

\begin{itemize}
    \item
\end{itemize}

\end{document}