\documentclass[12pt]{article}
\usepackage[top=1cm, bottom=2cm, right=1cm, left=1cm]{geometry}
\usepackage{amsfonts, amssymb, amsmath, hyperref}
\usepackage{graphicx}
\usepackage[T1, T2A]{fontenc}% T2A for Cyrillic font encoding
\usepackage[english, russian]{babel}
\usepackage[justification=centering]{caption}
\usepackage{wrapfig}
\usepackage{placeins}
\usepackage{subcaption}
\usepackage{multirow}
\usepackage{indentfirst}
\usepackage{needspace}
\usepackage{float}
\floatplacement{table}{H}

\begin{document}
\title{\textbf{Вопрос по выбору}\\ [2pt]{Исследование поглощения спектров паров йода.}}
\date{\today}
\author{Павлов Матвей, Редькин Денис}

\maketitle

\section*{Аннотация}
В работе измеряются спектры поглощения паров йода. По ним выделяются серии электронно-колебательных переходов и определяются параметры потенциала межъядерного взаимодействия в двухатомной молекуле йода.

\section*{Теоретические сведения}

Двухатомная молекула представляет собой квантовую систему, состоящую из двух атомных ядер и электронной оболочки. Теоретическое описание такой системы в рамках точной квантовой механики затруднено, поэтому на практике используются приближённые методы. Основным из них является адиабатическое приближение (приближение Борна–Оппенгеймера), основанное на существенном различии масс электронов и ядер.

В рамках этого приближения предполагается, что электроны, обладая значительно меньшей массой, мгновенно подстраиваются к медленному движению ядер. Волновая функция молекулы при этом представляется в виде произведения электронной и ядерной частей. Задача о движении молекулы сводится к последовательному решению двух задач: сначала — электронной, при фиксированных положениях ядер, а затем — ядерной в эффективном потенциале.

Решение электронной задачи для каждой конфигурации ядер $\{ \vec R_i \}$ определяет энергию электронной подсистемы $E_n^{el}$, зависящую от расстояния между ядрами. Для движения ядер вводится эффективный потенциал
\[
U_n^{\mathrm{eff}}(\{\vec R_i\}) = U_{\text{я}}(\{\vec R_i\}) + E_n^{el},
\]
который называется \textit{электронным термом} молекулы. Зависимость $U_n^{\mathrm{eff}}(\rho)$ от межъядерного расстояния $\rho$ называется потенциальной кривой соответствующего электронного состояния.

\subsection*{Колебательное движение ядер}

Для двухатомной молекулы движение ядер в первом приближении можно рассматривать как одномерное движение частицы с приведённой массой $\mu$ в потенциальной яме $U^{\mathrm{eff}}(\rho)$. Вблизи минимума потенциальной кривой колебательное движение может быть приближённо описано моделью гармонического осциллятора, для которого энергетические уровни имеют вид
\[
E_v = \hbar \omega_e \left( v + \frac{1}{2} \right), \qquad v = 0,1,2,\dots
\]
Здесь $\omega_e$ — собственная частота колебаний молекулы. Для гармонического осциллятора уровни энергии эквидистантны, а разрешённые переходы удовлетворяют правилу отбора $\Delta v = \pm 1$.

Однако реальное колебательное движение молекулы является ангармоническим, особенно при больших амплитудах колебаний. Для более точного описания используется потенциал Морзе, позволяющий учитывать ангармонизм и конечную энергию диссоциации молекулы:
\[
U^{\mathrm{eff}}(\rho) =
A\left(
e^{-2\alpha(\rho-\rho_0)} - 2e^{-\alpha(\rho-\rho_0)}
\right) + D,
\]
где $\rho_0$ — равновесное расстояние между ядрами, $A$ и $\alpha$ — параметры потенциала, $D$ — энергия диссоциации.

\begin{figure}[H]
\centering
\includegraphics[width=0.8\textwidth]{images/1.png}
\caption{Потенциал Морзе и колебательные уровни}
\label{fig:morse}
\end{figure}

Энергетический спектр колебательных уровней в потенциале Морзе имеет вид
\begin{equation}
E_n = D - A + \hbar \omega_{\text{кол}} \left( n + \frac{1}{2} \right)
- \frac{1}{4A}\left( \hbar \omega_{\text{кол}} \left( n + \frac{1}{2} \right) \right)^2,
\label{eq:morse_levels}
\end{equation}
где $\omega_{\text{кол}} = \alpha \sqrt{\dfrac{2A}{\mu}}$ — частота колебательного кванта.

Характерной особенностью данного спектра является уменьшение расстояния между соседними уровнями с ростом квантового числа $n$ и конечное число связанных состояний. Разности между соседними уровнями равны
\begin{equation}
\Delta E_n = E_{n+1} - E_n =
\hbar \omega_{\text{кол}}
\left(1 - \frac{\hbar \omega_{\text{кол}}}{2A}(n+1)\right),
\label{eq:deltaE}
\end{equation}
а вторые разности оказываются постоянными:
\begin{equation}
\Delta^2 E_n = -\frac{(\hbar \omega_{\text{кол}})^2}{2A}.
\label{eq:delta2E}
\end{equation}
Это позволяет по экспериментальному колебательному спектру определить параметры потенциала Морзе.

\subsection*{Вращательные уровни}

Вращательное движение двухатомной молекулы в первом приближении может быть описано моделью жёсткого ротатора. Энергия вращательных уровней имеет вид
\[
E_j = \frac{\hbar^2}{2I}j(j+1),
\qquad j = 0,1,2,\dots
\]
где $I$ — момент инерции молекулы. Разрешённые вращательные переходы удовлетворяют правилу отбора $\Delta j = \pm 1$.

В данной работе вклад вращательных уровней в спектр не рассматривается, поскольку разрешающая способность используемых приборов недостаточна для их количественного анализа.

\subsection*{Принцип Франка–Кондона}

Переходы между различными электронными термами молекулы могут сопровождаться электромагнитным излучением. Вероятность такого перехода определяется матричным элементом оператора дипольного момента. В рамках адиабатического приближения электронная и ядерная части волновой функции разделяются, и интенсивность электронно-колебательного перехода определяется перекрытием колебательных волновых функций начального и конечного состояний.

\begin{figure}[H]
\centering
\includegraphics[width=0.3\textwidth]{images/2.png}
\caption{Электронно-колебательные термы и разрешённые переходы}
\label{fig:franck_condon}
\end{figure}

Принцип Франка–Кондона формулируется следующим образом: наиболее вероятны те электронно-колебательные переходы, для которых интеграл перекрытия колебательных волновых функций
\[
\langle n | n' \rangle = \int \psi_n^{\mathrm{nuc}}(\rho)\,
\psi_{n'}^{\mathrm{nuc}}(\rho)\, d\rho
\]
имеет наибольшее значение. Квадрат этого интеграла определяет фактор Франка–Кондона и, следовательно, относительную интенсивность спектральных линий.

\section*{Экспериментальная установка}

\subsection*{Схема экспериментальной установки}
Установка для измерения спектров включает в себя оптическую скамью, источники излучения (газоразрядную ртутную лампу и лампу накаливания), кювету с парами йода, фокусирующую линзу и монохроматор с установленным на нём цифровым зеркальным фотоаппаратом. Источники излучения, кювета и линза устанавливаются на оптической скамье, которая жёстко соединена с корпусом монохроматора. Общая схема установки показана на рисунке \ref{fig:ustanovka}.

Излучение подаётся через коллимационную линзу на входную щель монохроматора, и пройдя через линзы и призму, фокусируется на плоскости светочувствительной матрицы фотоаппарата.

\subsection*{Устройство цифрового оптического спектрометра}
Для измерения оптических спектров в работе используется используется призменный монохроматор УМ-2 оборудованный цифровым фотоаппаратом Canon EOS 650D или Nikon D5300. Также возможен вариант с использованием монохроматора ИСП51. Цифровой фотоаппарат установлен вместо выходного окуляра монохроматора, так что изображение спектра формируется прямо на фоточувствительную матрицу. Использование цифрового фотоаппарата в качестве регистрирующего устройства позволяет повысить точность и чувствительность измерений, а также даёт возможность собрать большее количество данных и пронаблюдать более тонкие эффекты. При этом остаётся, по прежнему, возможность наблюдать спектр изучаемого излучения непосредственно глазом. Для этого достаточно опустить зеркало фотоаппарата в исходное положение.

\begin{figure}[h!]
  \centering
  \includegraphics[width=0.7\textwidth]{images/3_ust.png}  
  \caption{Схема установки регистрации оптических спектров}
  \label{fig:ustanovka}
\end{figure}

Фотоаппарат подключён к персональному компьютеру с помощью USB кабеля. C помощью поставляемой производителем фотоаппарата программы EOS Utility (Camera Control Pro - для Nikon D5300) можно осуществлять управление фотоаппаратом с компьютера и получать изображение формируемое на матрице фотоаппарата на экране монитора в режиме реального времени.

Получаемые фотографии сохраняются сразу на жёсткий диск компьютера. Регистрируемые данные записываются в файлы в raw-формате. В дальнейшем данные из этих файлов обрабатываются с помощью программы VisSpectra, которая позволяет выделить из этих raw-фотографий спектральные зависимости интенсивности излучения, а также провести спектральную калибровку.

\section*{Ход работы}

\subsection*{Съемка спектров}

Произведена съемка спектров ртутной лампы, лампы накаливания и кюветы с парами йоды. Съемка производилась с различающимися временами выдержки. На рисунках~\ref{fig:rtuti},~\ref{fig:nakal},~\ref{fig:cuveta} представлены примеры спектров.

\begin{figure}[H]
  \centering
  \includegraphics[width=0.45\textwidth]{./images/1_50.png}
  \includegraphics[width=0.4\textwidth]{./images/3.png}
  \caption{Спектр ртутной лампы для времени выдержки 0.02 секунды и 3 секунды.}
  \label{fig:rtuti}
\end{figure}

\begin{figure}[H]
  \centering
  \includegraphics[width=0.45\textwidth]{./images/1_5nakal.png}
  \includegraphics[width=0.45\textwidth]{./images/4nakal.png}
  \caption{Спектр лампы накаливания для времени выдержки 0.2 секунды и 4 секунды.}
  \label{fig:nakal}
\end{figure}

\begin{figure}[H]
  \centering
  \includegraphics[width=0.45\textwidth]{./images/1cuveta.png}
  \includegraphics[width=0.45\textwidth]{./images/6cuveta.png}
  \caption{Спектр кюветы с парами йода для времени выдержки 1 секунды и 6 секунд.}
  \label{fig:cuveta}
\end{figure}

\subsection*{Обработка спектров поглощения}

При помощи программы VisSpectra обработаем спектры и построим графики спектральной зависимости интенсивности излучения от длины волны 

Калибровку произведем на основе спектра ртутной лампы на короткой выдержке 0.02 секунды. А линии поглощения будем снимать по спектру кюветы с парами йода на выдержкой 6 секунд, так как он получился самым отчетливым. Спектр приведен на рисунке~\ref{fig:spectr_cuveta}

\begin{figure}[H]
  \centering
%   \includegraphics[width=0.8\textwidth]{./images/spectr_cuveta.png}
  \caption{Спектр кюветы с парами йода.}
  \label{fig:spectr_cuveta}
\end{figure}

Далее выделим серии линий поглощения. Для каждой серии в отдельный текстовый файл сохраним значения длин волн выделенных минимумов интенсивности. Пример выделения серий приведен на рисунке~\ref{fig:primer_spectra}.

\begin{figure}[H]
  \centering
  \includegraphics[width=0.8\textwidth]{./images/primer_spectra.png}
  \caption{Пример спектра поглощения паров йода с выделенными тремя различными сериями линий поглощения.}
  \label{fig:primer_spectra}
\end{figure}

\section*{Обработка данных}

Используя данные полученные из спектральных измерений вычислим на их основе следующие параметры молекулы йода: величину колебательного кванта для возбужденного электронного терма $\delta E_{vib.,ex}$ и, аналогично, для основного электронного терма $\delta E_{vib.,gr}$, оценим величину энергии между дном основного электронного терма и дном возбужденного электронного терма $\delta E_{el.,ex}$, оценим энергию, необходимую для диссоциации молекулы находящейся на возбужденном электрическом терме $\delta E_{dis.,ex}$. 
Определим энергию диссоциации молекулы на основном электронном терме $\delta E_{dis.,gr}$ и энергию возбуждения $\delta E_{a}$ отдельного атома йода. Для большей ясности, на рисунке~\ref{fig:coleb_spectr} эти значения указаны как интервалы относительно линий энергетического спектра колебательных состояний двухатомной молекулы. Для определения этих параметров обработаем данные полученных серий спектральных минимумов.

\begin{figure}[H]
  \centering
  \includegraphics[width=0.5\textwidth]{./images/coleb_spectr.png}
  \caption{Колебательный спектр двухатомной молекулы и обозначения энергетических интервалов.}
  \label{fig:coleb_spectr}
\end{figure}

\subsection*{График колебательного спектра}

Построим график колебательного спектра. Для этого

\subsection*{Параметры электронно-колебательных термов}


\section*{Вывод}
Таким образом, в ходе работы были получены следующие результаты:
\begin{itemize}
    \item
\end{itemize}

\end{document}