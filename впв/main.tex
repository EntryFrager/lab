\documentclass[12pt]{article}
\usepackage[top=1cm, bottom=2cm, right=1cm, left=1cm]{geometry}
\usepackage{amsfonts, amssymb, amsmath, hyperref}
\usepackage{graphicx}
\usepackage[T1, T2A]{fontenc}% T2A for Cyrillic font encoding
\usepackage[english, russian]{babel}
\usepackage[justification=centering]{caption}
\usepackage{wrapfig}
\usepackage{placeins}
\usepackage{subcaption}
\usepackage{multirow}
\usepackage{indentfirst}
\usepackage{needspace}
\usepackage{float}
\floatplacement{table}{H}

\begin{document}
\title{\textbf{Вопрос по выбору}\\ [2pt]{Исследование поглощения спектров паров йода.}}
\date{\today}
\author{Павлов Матвей, Редькин Денис}

\maketitle

\section*{Аннотация}
В работе измеряются спектры поглощения паров йода. По ним выделяются серии электронно-колебательных переходов и определяются параметры потенциала межъядерного взаимодействия в двухатомной молекуле йода.

\section*{Теоретические сведения}

Двухатомная молекула представляет собой квантовую систему, состоящую из двух атомных ядер и электронной оболочки. Теоретическое описание такой системы в рамках точной квантовой механики затруднено, поэтому на практике используются приближённые методы. Основным из них является адиабатическое приближение (приближение Борна–Оппенгеймера), основанное на существенном различии масс электронов и ядер.

В рамках этого приближения предполагается, что электроны, обладая значительно меньшей массой, мгновенно подстраиваются к медленному движению ядер. Волновая функция молекулы при этом представляется в виде произведения электронной и ядерной частей. Задача о движении молекулы сводится к последовательному решению двух задач: сначала — электронной, при фиксированных положениях ядер, а затем — ядерной в эффективном потенциале.

Решение электронной задачи для каждой конфигурации ядер $\{ \vec R_i \}$ определяет энергию электронной подсистемы $E_n^{el}$, зависящую от расстояния между ядрами. Для движения ядер вводится эффективный потенциал
\[
U_n^{\mathrm{eff}}(\{\vec R_i\}) = U_{\text{я}}(\{\vec R_i\}) + E_n^{el},
\]
который называется \textit{электронным термом} молекулы. Зависимость $U_n^{\mathrm{eff}}(\rho)$ от межъядерного расстояния $\rho$ называется потенциальной кривой соответствующего электронного состояния.

\subsection*{Колебательное движение ядер}

Для двухатомной молекулы движение ядер в первом приближении можно рассматривать как одномерное движение частицы с приведённой массой $\mu$ в потенциальной яме $U^{\mathrm{eff}}(\rho)$. Вблизи минимума потенциальной кривой колебательное движение может быть приближённо описано моделью гармонического осциллятора, для которого энергетические уровни имеют вид
\[
E_v = \hbar \omega_e \left( v + \frac{1}{2} \right), \qquad v = 0,1,2,\dots
\]
Здесь $\omega_e$ — собственная частота колебаний молекулы. Для гармонического осциллятора уровни энергии эквидистантны, а разрешённые переходы удовлетворяют правилу отбора $\Delta v = \pm 1$.

Однако реальное колебательное движение молекулы является ангармоническим, особенно при больших амплитудах колебаний. Для более точного описания используется потенциал Морзе, позволяющий учитывать ангармонизм и конечную энергию диссоциации молекулы:
\[
U^{\mathrm{eff}}(\rho) =
A\left(
e^{-2\alpha(\rho-\rho_0)} - 2e^{-\alpha(\rho-\rho_0)}
\right) + D,
\]
где $\rho_0$ — равновесное расстояние между ядрами, $A$ и $\alpha$ — параметры потенциала, $D$ — энергия диссоциации.

\begin{figure}[H]
\centering
\includegraphics[width=0.8\textwidth]{images/1.png}
\caption{Потенциал Морзе и колебательные уровни}
\label{fig:morse}
\end{figure}

Энергетический спектр колебательных уровней в потенциале Морзе имеет вид
\begin{equation}
E_n = D - A + \hbar \omega_{\text{кол}} \left( n + \frac{1}{2} \right)
- \frac{1}{4A}\left( \hbar \omega_{\text{кол}} \left( n + \frac{1}{2} \right) \right)^2,
\label{eq:morse_levels}
\end{equation}
где $\omega_{\text{кол}} = \alpha \sqrt{\dfrac{2A}{\mu}}$ — частота колебательного кванта.

Характерной особенностью данного спектра является уменьшение расстояния между соседними уровнями с ростом квантового числа $n$ и конечное число связанных состояний. Разности между соседними уровнями равны
\begin{equation}
\Delta E_n = E_{n+1} - E_n =
\hbar \omega_{\text{кол}}
\left(1 - \frac{\hbar \omega_{\text{кол}}}{2A}(n+1)\right),
\label{eq:DeltaE}
\end{equation}
а вторые разности оказываются постоянными:
\begin{equation}
\Delta^2 E_n = -\frac{(\hbar \omega_{\text{кол}})^2}{2A}.
\label{eq:Delta2E}
\end{equation}
Это позволяет по экспериментальному колебательному спектру определить параметры потенциала Морзе.

\subsection*{Вращательные уровни}

Вращательное движение двухатомной молекулы в первом приближении может быть описано моделью жёсткого ротатора. Энергия вращательных уровней имеет вид
\[
E_j = \frac{\hbar^2}{2I}j(j+1),
\qquad j = 0,1,2,\dots
\]
где $I$ — момент инерции молекулы. Разрешённые вращательные переходы удовлетворяют правилу отбора $\Delta j = \pm 1$.

В данной работе вклад вращательных уровней в спектр не рассматривается, поскольку разрешающая способность используемых приборов недостаточна для их количественного анализа.

\subsection*{Принцип Франка–Кондона}

Переходы между различными электронными термами молекулы могут сопровождаться электромагнитным излучением. Вероятность такого перехода определяется матричным элементом оператора дипольного момента. В рамках адиабатического приближения электронная и ядерная части волновой функции разделяются, и интенсивность электронно-колебательного перехода определяется перекрытием колебательных волновых функций начального и конечного состояний.

\begin{figure}[H]
\centering
\includegraphics[width=0.3\textwidth]{images/2.png}
\caption{Электронно-колебательные термы и разрешённые переходы}
\label{fig:franck_condon}
\end{figure}

Принцип Франка–Кондона формулируется следующим образом: наиболее вероятны те электронно-колебательные переходы, для которых интеграл перекрытия колебательных волновых функций
\[
\langle n | n' \rangle = \int \psi_n^{\mathrm{nuc}}(\rho)\,
\psi_{n'}^{\mathrm{nuc}}(\rho)\, d\rho
\]
имеет наибольшее значение. Квадрат этого интеграла определяет фактор Франка–Кондона и, следовательно, относительную интенсивность спектральных линий.

\section*{Экспериментальная установка}

\subsection*{Схема экспериментальной установки}
Установка для измерения спектров включает в себя оптическую скамью, источники излучения (газоразрядную ртутную лампу и лампу накаливания), кювету с парами йода, фокусирующую линзу и монохроматор с установленным на нём цифровым зеркальным фотоаппаратом. Источники излучения, кювета и линза устанавливаются на оптической скамье, которая жёстко соединена с корпусом монохроматора. Общая схема установки показана на рисунке \ref{fig:ustanovka}.

Излучение подаётся через коллимационную линзу на входную щель монохроматора, и пройдя через линзы и призму, фокусируется на плоскости светочувствительной матрицы фотоаппарата.

\subsection*{Устройство цифрового оптического спектрометра}
Для измерения оптических спектров в работе используется используется призменный монохроматор УМ-2 оборудованный цифровым фотоаппаратом Canon EOS 650D или Nikon D5300. Также возможен вариант с использованием монохроматора ИСП51. Цифровой фотоаппарат установлен вместо выходного окуляра монохроматора, так что изображение спектра формируется прямо на фоточувствительную матрицу. Использование цифрового фотоаппарата в качестве регистрирующего устройства позволяет повысить точность и чувствительность измерений, а также даёт возможность собрать большее количество данных и пронаблюдать более тонкие эффекты. При этом остаётся, по прежнему, возможность наблюдать спектр изучаемого излучения непосредственно глазом. Для этого достаточно опустить зеркало фотоаппарата в исходное положение.

\begin{figure}[h!]
  \centering
  \includegraphics[width=0.7\textwidth]{images/3_ust.png}  
  \caption{Схема установки регистрации оптических спектров}
  \label{fig:ustanovka}
\end{figure}

Фотоаппарат подключён к персональному компьютеру с помощью USB кабеля. C помощью поставляемой производителем фотоаппарата программы EOS Utility (Camera Control Pro - для Nikon D5300) можно осуществлять управление фотоаппаратом с компьютера и получать изображение формируемое на матрице фотоаппарата на экране монитора в режиме реального времени.

Получаемые фотографии сохраняются сразу на жёсткий диск компьютера. Регистрируемые данные записываются в файлы в raw-формате. В дальнейшем данные из этих файлов обрабатываются с помощью программы VisSpectra, которая позволяет выделить из этих raw-фотографий спектральные зависимости интенсивности излучения, а также провести спектральную калибровку.

\section*{Ход работы}

\subsection*{Съемка спектров}

Произведена съемка спектров ртутной лампы, лампы накаливания и кюветы с парами йоды. Съемка производилась с различающимися временами выдержки. На рисунках~\ref{fig:rtuti},~\ref{fig:nakal},~\ref{fig:cuveta} представлены примеры спектров.

\begin{figure}[H]
  \centering
  \includegraphics[width=0.45\textwidth]{./images/1_50.png}
  \includegraphics[width=0.4\textwidth]{./images/3.png}
  \caption{Спектр ртутной лампы для времени выдержки 0.02 секунды и 3 секунды.}
  \label{fig:rtuti}
\end{figure}

\begin{figure}[H]
  \centering
  \includegraphics[width=0.45\textwidth]{./images/1_5nakal.png}
  \includegraphics[width=0.45\textwidth]{./images/4nakal.png}
  \caption{Спектр лампы накаливания для времени выдержки 0.2 секунды и 4 секунды.}
  \label{fig:nakal}
\end{figure}

\begin{figure}[H]
  \centering
  \includegraphics[width=0.45\textwidth]{./images/1cuveta.png}
  \includegraphics[width=0.45\textwidth]{./images/6cuveta.png}
  \caption{Спектр кюветы с парами йода для времени выдержки 1 секунды и 6 секунд.}
  \label{fig:cuveta}
\end{figure}

\subsection*{Обработка спектров поглощения}

При помощи программы VisSpectra обработаем спектры и построим графики спектральной зависимости интенсивности излучения от длины волны 

На основе спектра ртутной лампы на короткой выдержке 0.02 секунды установим уровень темного фона. Калибровку произведем на основе спектра ртутной лампы на выдержке 1 секунда. А линии поглощения будем снимать по спектру кюветы с парами йода на выдержкой 6 секунд, так как он получился самым отчетливым.

Далее выделим серии линий поглощения. Для каждой серии в отдельный текстовый файл сохраним значения длин волн выделенных минимумов интенсивности. Пример выделения серий приведен на рисунке~\ref{fig:spectr_cuveta}.

\begin{figure}[H]
  \centering
  \includegraphics[width=0.8\textwidth]{./images/spectr_cuveta.png}
  \caption{Спектр кюветы с парами йода.}
  \label{fig:spectr_cuveta}
\end{figure}

\newpage

\section*{Обработка данных}

Используя данные полученные из спектральных измерений вычислим на их основе следующие параметры молекулы йода: величину колебательного кванта для возбужденного электронного терма $\Delta E_{vib.,ex}$ и, аналогично, для основного электронного терма $\Delta E_{vib.,gr}$, оценим величину энергии между дном основного электронного терма и дном возбужденного электронного терма $\Delta E_{el.,ex}$, оценим энергию, необходимую для диссоциации молекулы находящейся на возбужденном электрическом терме $\Delta E_{dis.,ex}$. 
Определим энергию диссоциации молекулы на основном электронном терме $\Delta E_{dis.,gr}$ и энергию возбуждения $\Delta E_{a}$ отдельного атома йода. Для большей ясности, на рисунке~\ref{fig:coleb_spectr} эти значения указаны как интервалы относительно линий энергетического спектра колебательных состояний двухатомной молекулы. Для определения этих параметров обработаем данные полученных серий спектральных минимумов.

\begin{figure}[H]
  \centering
  \includegraphics[width=0.5\textwidth]{./images/coleb_spectr.png}
  \caption{Колебательный спектр двухатомной молекулы и обозначения энергетических интервалов.}
  \label{fig:coleb_spectr}
\end{figure}

Для каждой серии вычислим значения первых разностей уровней энергии. Номера серий $m$ будем назначать так, чтобы большему номеру серии $m$ соответствовала меньшая минимальная энергия линии поглощения в серии ($E_1(0) > E_2(0) > E_3(0) > ...$). На основе этих данных построим график зависимости значений первых разностей энергии уровней от номера этого значения в серии.

\[\Delta E_{n_m} = E_m(n+1) - E_m(n)\]

\[\sigma_{E_{n_m}} = \sqrt{\sigma_{E_m{n+1}}^2 + \sigma_{E_m{n}}^2}\]


\begin{figure}[H]
  \centering
  \includegraphics[width=1.0\textwidth]{./images/pervie_raznosti.png}
  \caption{Графики серии переходов и первых разностей.}
  \label{fig:pervie_raznosti}
\end{figure}

Далее, подберем значения индексов смещения линий поглощения для каждой серии, такие, чтобы все зависимости укладывались в одну общую зависимость, то есть лежали на одной прямой. Построим график зависимостей энергий для каждой серии с учетом разностей.

\[
n'_m = n + \Delta n_m
\]

Подбором получаем:

\begin{table}[H]
    \centering
    \begin{tabular}{|c|c|}
        \hline
        Серия & $\Delta n_m$ \\ \hline
        1 & 1 \\ \hline
        2 & 1 \\ \hline
        3 & -4 \\ \hline
        4 & -5 \\ \hline
        5 & -5 \\ \hline
    \end{tabular}
    \caption{Индексы смещения линий поглощения для каждой серии}
\end{table}

Аппроксимируем зависимость:

\[
\Delta E(n') = b + k(n' + 1)
\]

Сравнивая ее с формулой Морзе \ref{eq:DeltaE}, получаем:

\[
k = - \frac{(\hbar w_{vib.,ex})^2}{2A} = (- 2.56 \pm 0.03) \cdot 10^{-4} \text{ эВ}
\]

\[
b = \hbar w_{vib.,ex} = 0.01319 \pm 0.00004 \text{ эВ}
\]

\begin{figure}[H]
  \centering
  \includegraphics[width=1.0\textwidth]{./images/raznosti_smechenie.png}
  \caption{Графики серии переходов и первых разностей со смещением индексов.}
  \label{fig:pervie_raznosti}
\end{figure}

Убедимся что в области пересечения графиков для различных серий, разница между энергиями остаётся примерно одинаковой. Вычислим средние значения такой разницы для всех пар соседних серий. Построим зависимость этой величин от номера серии - это будет график первых разностей энергий для основного электронного терма. Из этого графика определим квант колебательного движения основного электронного терма – $\Delta E_{vib.,gr}$.

$\Delta E_{vib.,gr}$ является средняя энергия всех этих разностей.

$$
\Delta E_{vib.,gr} = (26.4 \pm 0.1) \cdot 10^{-3} \text{ эВ}
$$

\begin{figure}[h!]
  \centering
  \includegraphics[width=0.8\textwidth]{./images/1_graph.png}
  \label{fig:1_graph}
\end{figure}

Для каждой пар серий посчитаем среднее и построим графики этой зависимости от номера серии.

\begin{figure}[H]
  \centering
  \includegraphics[width=0.8\textwidth]{./images/srednee.png}
  \caption{График средней разности энергий для каждых соседних пар}
\end{figure}

Для определения погрешности наклона определим $k_{min}, k_{max}$ проходящее через областей погрешности точек.

\[
\Delta^2 E_{gr} = (5.1 \pm 0.8) \cdot 10^{-4} \text{эВ}
\]

По наклону прямой графика первых разностей колебательных уровней основного терма определим значение вторых разниц, соответствующих формуле \ref{eq:Delta2E}.

\[
\Delta^2 E_{ex} = (2.56 \pm 0.03) \cdot 10^{-4} \text{ эВ}
\]

Из графика значений первых разностей для возбужденного электронного терма для всех серий, с учетом ранее найденных индексов смещения, определим соответствующий колебательный квант: $\Delta E_{vib.,ex}$.

\[
\Delta E_{vib.,ex} = (13.19 \pm 0.04) \cdot 10^{-3} \text{ эВ}
\]

Найдем величину энергии между дном основного электронного терма и дном возбужденного электронного терма $\Delta E_{el.,ex}$. Для этого нужно продолжить параболой 1 серию (серия переходов из 0 невозбужденного терма в возбужденные термы).

\[
E^{(1)}_n = W_2 + W_1 \left(n+\frac{1}{2} \right) + W_0 \left(n+\frac{1}{2} \right)^2
\]

\[
W_0 = - 1.26 \cdot 10^{-4} \text{ эВ} \quad W_1 = 1.32 \cdot 10^{-2} \text{ эВ} \quad W_2 = 2.12 \text{ эВ}
\]

\[
\varepsilon_W < 1 \%
\]

$W_2$ является разницей между энергиями $n=0$ невозбужденного терма и дном основного электронного терма.

$$
\Delta E_{el, ex} = W_2 + \frac{\hbar \omega_e}{2} = W_2 + \frac{\Delta E_{v i b ., g r}}2 = (2.13 \pm 0.01) \text{эВ}
$$

\begin{figure}[H]
  \centering
  \includegraphics[width=1.0\textwidth]{./images/parabola.png}
  \caption{Приближение 1 серии параболой.}
  \label{fig:parabola}
\end{figure}

По формуле \ref{eq:Delta2E} вычислим параметры $\alpha_{gr}$ и $\alpha_{ex}$ потенциала Морзе, аппроксимирующего потенциальную кривую основного и возбужденного электронного термов. Для $\mathrm{I}_2$ приведенная масса $\mu = 63.5 \, \text{а.м.}$.

\[
\alpha_{gr} = \frac{\sqrt{-\mu \Delta^2 E_{gr} }} {\hbar} = (3.8 \pm 0.1) \cdot 10^{10} \text{ м}^{-1}
\]

\[
\alpha_{ex} = \frac{\sqrt{-\mu \Delta^2 E_{ex} }} {\hbar} = (1.90 \pm 0.02) \cdot 10^{10} \text{ м}^{-1}
\]

Оценим полное число $N_{gr}$ и $N_{ex}$ колебательных уровней основного и возбужденного электронных термов. Выведем необходимую формулу из вида энергетического спектра потенциала Морзе \ref{eq:morse_levels}.

\[
E_N = D - A + \hbar \omega_e \left(N + \frac{1}{2} \right) - \frac{(\hbar \omega_e)^2}{4A}\left(N+\frac{1}{2} \right)^2 \approx D \Rightarrow \\
\]

\[
N + \frac{1}{2} = \dfrac{\frac{\hbar \omega_e}{2}}{\frac{(\hbar \omega_e)^2}{4A}} = 2 \frac{A}{\hbar \omega_e} = \frac{\Delta E}{\Delta^2 E} \\
\]

\[
N_{gr} = 52 \pm 15
\]

\[
N_{ex} = 52 \pm 3
\]

Определим энергию диссоциации основного $\Delta E_{dis.,ex}$ и возбужденного $\Delta E_{dis.,ex}$ электронных термов, а также энергию возбуждения атома йода $\Delta E_a$.

\[
\Delta E_{dis.} = D - (D-A) = A = \frac12 \frac{(\Delta E)^2}{\Delta^2 E}
\]

\[
\Delta E_{dis.,gr} = 0.68 \pm 0.11 \text{ эВ}
\]

\[
\Delta E_{dis.,ex} = 0.340 \pm 0.004 \text{ эВ}
\]

\[
\Delta E_a = \Delta E_{el.,ex} + \Delta E_{dis.,ex} - \Delta E_{dis.,gr} = 1.79 \pm 0.15 \text{ эВ}
\]

\section*{Вывод}
Таким образом, в ходе работы был изучен спектр поглощения паров йода. Оценены параметры электронно-колебательных термов:
\begin{itemize}
  \item Определен квант колебательного движения основного электронного терма:
  \[
  \Delta E_{vib.,gr} = (26.4 \pm 0.1) \cdot 10^{-3} \text{ эВ}
  \] 
  Табличное значение:
  \[
  \Delta E^{\text{табл.}}_{v i b ., g r} = 26.5 \cdot 10^{-3} \text{ эВ}
  \]
  \item Для возбужденного электронного терма определен соответствующий колебательный квант:
  \[
  \Delta E_{vib, ex} = (13.19 \pm 0.04) \cdot 10^{-3} \text{ эВ}
  \]
  Табличное значение:
  \[
  \Delta E^{\text{табл.}}_{v i b ., ex} = 14.9 \cdot 10^{-3} \text{ эВ}
  \]
  \item Найдена величина энергии между дном основного электронного терма и дном возбужденного электронного терма:
  \[
  \Delta E_{el, ex} = (2.13 \pm 0.01) \text{ эВ}
  \]
  \item Вычислены энергии диссоциации основного и возбужденного электронных термов, а также энергия возбуждения атома йода:
  \[
  \Delta E_{d i s ., gr} = (0.68 \pm 0.11) \text{ эВ}
  \]
  \[
  \Delta E_{d i s ., ex} = (0.340 \pm 0.004) \text{ эВ}
  \]
  \[
  \Delta E_a =  (1.79 \pm 0.15) \text{ эВ}
  \]
  Табличное значение:
  \[
  \Delta E^{\text{табл.}}_{a} = 0.94 \text{ эВ}
  \]
\end{itemize}

\end{document}