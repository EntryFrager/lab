\documentclass[12pt]{article}
\usepackage[top=1cm, bottom=2cm, right=1cm, left=1cm]{geometry}
\usepackage{amsfonts, amssymb, amsmath, hyperref}
\usepackage{graphicx}
\usepackage[T1, T2A]{fontenc}% T2A for Cyrillic font encoding
\usepackage[english, russian]{babel}
\usepackage[justification=centering]{caption}
\usepackage{wrapfig}
\usepackage{placeins}
\usepackage{subcaption}
\usepackage{multirow}
\usepackage{indentfirst}
\usepackage{needspace}
\usepackage{float}
\floatplacement{table}{H}

\begin{document}
\title{\textbf{Лабораторная работа 5.4.1}\\ [2pt]{Определение энергии $\alpha$-частиц по величине их пробега в воздухе}}
\date{\today}
\author{Павлов Матвей, Елисеев Данил}

\maketitle

\section*{Аннотация}
    Измерить пробег $\alpha$ частиц в воздухе тремя способами: с помощью торцевого счетчика Гейгера, сцинтилляционного счетчика и ионизационной камеры, --- по полученным данным определить энергию частиц
	
\section*{Теоретические сведения}

Энергетическое условие $\alpha$-распада имеет вид
\[
\Delta E = E_{A-4,Z-2} + E_\alpha - E_{A,Z} > 0,
\]
где $E_{A,Z}$~--- энергия связи исходного ядра, $E_{A-4,Z-2}$~--- дочернего, а $E_\alpha \approx 28 \,\text{МэВ}$~--- энергия связи $\alpha$-частицы. Для средних ядер, где энергия связи на нуклон достигает примерно $8 \,\text{МэВ}$, распад невозможен, и он осуществляется только у тяжёлых ядер с $Z > 83$.

Потенциальная энергия $\alpha$-частицы вне ядра определяется кулоновским отталкиванием
\[
U(r) = \frac{2(Z-2)e^2}{r},
\]
внутри ядра потенциал практически постоянен, образуя потенциальную яму (рис.~\ref{fig:1}). Классически $\alpha$-частица не могла бы покинуть ядро, однако в квантовой механике вероятность её нахождения вне ядра отлична от нуля, что объясняет возможность туннелирования.


\begin{wrapfigure}{l}{0.3\textwidth}
    \centering
    \includegraphics[width=0.3\textwidth]{images/1.png}
    \caption{Качественный вид потенциальной энергии $\alpha$-частицы как функции расстояния от центра ядра.}
    \label{fig:1}
    \vspace{-2cm}
\end{wrapfigure}

Вероятность преодоления барьера определяется проницаемостью
\[
\ln D = -\frac{2}{\hbar} \int\limits_{R_y}^{R_p} \sqrt{2m_\alpha \bigl(U(r)-E\bigr)} \, dr,
\]
где $R_y$~--- радиус ядра, а $R_p$~--- точка поворота, определяемая из условия $E = 2(Z-2)e^2/R_p$. При $E \ll U(R_y)$ этот интеграл аппроксимируется выражением
\[
\ln D \simeq -A \left(\frac{B}{\sqrt{E}} + C\right).
\]
Так как период полураспада $T_{1/2}$ обратно пропорционален проницаемости пробега, мы приходим к эмпирическому закону Гейгера--Нэттола, связывающему период полураспада ядра и энергию вылетающей $\alpha$-частицы: $\lg T_{1/2} = \frac{a}{\sqrt{E}} + b$, где величины $a$ и $b$ выражаются через заряд ядра как $a \simeq 1{,}6Z, b \simeq -1{,}6Z^{2/3} - 21{,}4$.

Вывод этого закона основывается на зависимости проницаемости барьера от энергии: так как вероятность выхода частицы из ядра пропорциональна $D$, а константа распада $\lambda = \nu D$,
где число попыток $\nu \simeq \frac{\hbar}{2m_\alpha R_y^2}$, то период полураспада $T_{1/2} = \frac{\ln 2}{\lambda}$
принимает вид, совпадающий с указанной зависимостью.

Длина пробега частицы с зарядом $z$ и массой $M$ в веществе с атомным номером $Z$ вычисляется по формуле:

\[
R_{zM}(E_0, E_1) = -\int_{E_0}^{E_1} \frac{dE}{dE/dx} = \frac{m}{2\pi e^4 z^2 n Z} \int_{E_1}^{E_0} \frac{v^2  dE}{\ln(2mv^2 / I)}
\]

С учетом $dE = M v dv$:
\[
R_{zM} = \frac{m M}{2\pi e^4 z^2 n Z} \int_{v_1}^{v_0} \frac{v^3  dv}{\ln(2mv^2 / I)}
\]

\begin{wrapfigure}{l}{0.3\textwidth}
    \centering
    \includegraphics[width=0.3\textwidth]{images/6.png}
    \caption{Зависимость числа $\alpha$-частиц от глубины их проникновения в вещество.}
    \label{fig:4}
\end{wrapfigure}

При пренебрежении логарифмической зависимостью: $R \propto \frac{M}{z^2} v_0^4 \propto E^2 $

Для $\alpha$-частиц в воздухе в диапазоне 4-9 МэВ используется эмпирическая формула: $R = 0{,}32 \cdot E^{3/2} $
где $R$ - в см, $E$ - в МэВ. Пробеги составляют несколько сантиметров.

Из-за рассеяния и статистического характера потерь энергии пробеги частиц с одинаковой начальной энергией различаются. Кривая числа частиц от глубины показывает постоянство числа частиц на малых глубинах и постепенное уменьшение в конце. Большая часть частиц останавливается в узкой области около среднего пробега $R_{\text{ср}}$. Экстраполированный пробег $R_{\text{э}}$ определяется пересечением касательной к кривой $N(x)$ в точке $R_{\text{ср}}$ с осью $x$.

В эксперименте с пучками конечной расходимости брэгговский пик смещается и размывается, поэтому для оценки пробега используется экстраполированный пробег $R_{\text{э}}$.

\section*{Экспериментальная установка}

В данной работе пробег $\alpha$-частиц в воздухе определяется тремя способами:
	\begin{enumerate}
		\item
			Счетчик Гейгера --- измеряет зависимость числа частиц от расстояния. Состоит из вакуумной камеры с источником, подвижного счётчика с слюдяным окном, коллиматора и электроники для регистрации;
		\item
			Cцинтилляционный счетчик --- регистрирует вспышки света от люминофора при изменении давления. Включает герметичную камеру, люминофорную пластину, фотоумножитель и систему вакуумирования;
		\item
			Ионизационная камера --- измеряет общий ток ионизации. Состоит из сферической камеры с электродами, источника напряжения 300 В, вакуумной системы и электрометра. Пробег определяют по зависимости тока от давления.
	\end{enumerate} 
	
	\begin{figure}[h!]
		\centering
        \includegraphics[width=0.9\textwidth]{images/5.png}
        \caption{Экспериментальные установки.}
        \label{fig:4}
	\end{figure}
	
	В качестве источника $\alpha$-частиц используется ${239}$Pu с периодом полураспада $T_{1/2} = 2,44 \cdot 10^4$ лет. Альфа-частицы, испускаемые ${239}$Pu состоят их трех моноэнергетических групп, различие между которыми лежит в пределах 50 кэВ. При той точности, которая достигается в наших опытах, их можно считать совпадающими по энергии, равной 5,15 МэВ.
	
\newpage

\section*{Ход работы}

\subsection*{Исследование пробега $\alpha$-частиц с помощью счетчика Гейгера}

Проведем измерения зависимости скорости счета N от расстояния x между источником и счетчиком.

Методика измерения счета и его погрешности стандартные: мы считаем число $ N' $ зарегистрированных частиц со статистической погрешностью $ \sigma_{N'}  = \sqrt{N'} $ и время регистрации $ t $ (Погрешность измерения $t$ мала по сравнению со статистической погрешностью, поэтому ее не будем учитывать), откуда получаем скорость счета $ N = N'/t $ и ее погрешность 
		
\begin{equation}\label{}
\sigma_{N} = N \cdot \dfrac{\sqrt{N'}}{N'} = \dfrac{\sqrt{N'}}{t}
\end{equation}

Погрешность для $ l $ оценим как $ \sigma_l = 0,5 $ мм --- погрешность цены деления. Результаты измерений величин и их погрешностей занесем в таблицу и построим график.

\begin{table}[H]
    \centering
    \begin{tabular}{|c|c|c|c|c|c|}
        \hline
        {$x$, мм} & {$\sigma_x$, мм} & {$t$, c} & {$N'$, отсчетов} & {$N$, отсчетов/с} & {$\sigma_N$, отсчетов/с} \\ \hline
        0,00      & 0,13         & 60,127   & 1020    & 16,96  &    0,53          \\ \cline{1-6}
        1,00      & 0,13         & 59,721   & 1008    & 16,88  &    0,53         \\ \cline{1-6}
        2,00      & 0,13         & 59,999   & 1032    & 17,20  &    0,54          \\ \cline{1-6}
        3,00      & 0,13         & 59,595   &  989    & 16,60  &    0,53          \\ \cline{1-6}
        4,00      & 0,13         & 59,883   &  931    & 15,55  &    0,51          \\ \cline{1-6}
        5,00      & 0,13         & 60,235   &  959    & 15,92  &    0,51          \\ \cline{1-6}
        6,00      & 0,13         & 59,852   &  893    & 14,92  &    0,50          \\ \cline{1-6}
        7,00      & 0,13         & 60,082   &  633    & 10,53  &    0,42          \\ \cline{1-6}
        7,25      & 0,13         & 60,263   &  453    &  7,52  &    0,35          \\ \cline{1-6}
        7,50      & 0,13         & 59,895   &  371    &  6,19  &    0,32          \\ \cline{1-6}
        7,75      & 0,13         & 59,884   &  273    &  4,56  &    0,28          \\ \cline{1-6}
        8,00      & 0,13         & 60,208   &  159    &  2,64  &    0,21          \\ \cline{1-6}
        8,25      & 0,13         & 60,204   &   71    &  1,18  &    0,14          \\ \cline{1-6}
        8,50      & 0,13         & 60,175   &   57    &  0,95  &    0,13          \\ \cline{1-6}
        8,75      & 0,13         & 60,199   &   40    &  0,66  &    0,11          \\ \cline{1-6}
        9,00      & 0,13         & 59,926   &   43    &  0,72  &    0,11          \\ \cline{1-6}
    \end{tabular}
    \caption{Счетчик Гейгера}
\end{table}

Также будем учитывать длинну коллиматора $d = 10$ мм, считаем, что суммарная погрешность $x+d$ будет $\sigma_{x+d} \approx \sigma_x$. Далее для удобства будем писать $x$ вместо $x + d$

Представим результаты эксперимента на графике в координатах $N,~ x$ и определим средний и экстраполированный пробег $\alpha$-частиц в см и г/см$^2$.

Плотность воздуха в тот день была равна $\rho \approx 1,178 \frac{\text{кг}}{\text{м}^3} = 1,178 \cdot 10^{-3}\frac{\text{г}}{\text{см}^3}$

Среднюю длину пробега оценим как полусумму крайних точек, которые мы использовали для аппроксимации наклонной кривой $ R_{\text{ср}}\backsimeq \frac{x_{1} + x_{2}}{2} = \frac{16 + 18,5}{2}$ мм $= 17,25 \pm 0,13  $ мм $ \Rightarrow R'_{\text{ср}} = \rho R_{\text{ср}} = 2,03 \pm 0,02)\cdot10^{-3}\frac{\text{г}}{\text{см}^2}$

Экстрапролированный пробег оценим как:
\[
    R_{\text{э}} = \dfrac{b}{a} = 18.56 \pm0.66~ \text{мм} ~ \Rightarrow R'_{\text{э}} = \rho R_{\text{э}} = (2,19 \pm 0,08)\cdot10^{-3}\frac{\text{г}}{\text{см}^2}
\]

И погрешность $R_{\text{э}}$ оценена как

\[
    \sigma_{R_{\text{э}}} = \sqrt{\left(\frac{\partial R_{\text{э}}}{\partial a}\sigma_{a}\right)^2 + \left(\frac{\partial R_{\text{э}}}{\partial b}\sigma_{b}\right)^2}
\]

Энергию таких $\alpha$-частиц можно оценить по эмпирической формуле 
		
\[
    R = 0,32 E^{3/2} \Rightarrow E_{\text{э}} = 3,23\pm0,77~ \text{МэВ},  \quad E_{\text{ср}} = 3,07 \pm 0,15~\text{МэВ}
\]
\begin{figure}[H]
    \centering
    \includegraphics[width=0.9\textwidth]{images/graph_1.png}
    \caption{Зависимость скорость счета частиц от расстояния между источником и счетчиком, аппроксимированная двумя прямыми}
    \label{fig:4}
\end{figure}

\subsection*{Определение пробега $\alpha$-частиц с помощью сцинтилляционного счетчика и ионизационной камеры}

Сцинтилляционный счетчик и ионизационная камера подключены к одному насосу, поэтому измерения зависимости тока и числа отсчетов от давления производились одновременно.

Атмосферное давление: $P_{a} = 740$ мм рт ст.
Температура $ T = 292 $ К. 

Снимем зависимость тока и числа отсчетов от давления.
Погрешность давления оценим как цену деления --- $ \sigma_P = 5 $ мм рт ст, погрешность $ \sigma_I = 3 $ пФ. Результаты измерения занесем в таблицу и построим график. Погрешность числа отсчетов также равна $\sigma_N = \sqrt{N}$, будем округлять ее до целого числа.


\begin{table}[H]
\centering
\begin{tabular}{|c|c|c|c|c|c|c|}
\hline
$P'$, мм рт. ст. & $P$, мм рт. ст. & $\sigma_P$, мм рт. ст. & $I$, пА & $\sigma_I$, пА & $N$, число отсчетов & $\sigma_N$ \\ \hline
725 & 15 & 5 & 5 & 3 & 3343 & 58 \\ \cline{1-7}
700 & 40 & 5 & 47 & 3 & 3215 & 57 \\ \cline{1-7}
675 & 65 & 5 & 85 & 3 & 2633 & 51 \\ \cline{1-7}
650 & 90 & 5 & 120 & 3 & 2320 & 48 \\ \cline{1-7}
625 & 115 & 5 & 155 & 3 & 1773 & 42 \\ \cline{1-7}
600 & 140 & 5 & 193 & 3 & 1308 & 36 \\ \cline{1-7}
575 & 165 & 5 & 231 & 3 & 744 & 27 \\ \cline{1-7}
550 & 190 & 5 & 270 & 3 & 392 & 20 \\ \cline{1-7}
525 & 215 & 5 & 306 & 3 & 218 & 15 \\ \cline{1-7}
500 & 240 & 5 & 346 & 3 & 126 & 11 \\ \cline{1-7}
475 & 265 & 5 & 386 & 3 & 65 & 8 \\ \cline{1-7}
450 & 290 & 5 & 426 & 3 & 36 & 6 \\ \cline{1-7}
425 & 315 & 5 & 468 & 3 & 34 & 6 \\ \cline{1-7}
400 & 340 & 5 & 509 & 3 & 5 & 2 \\ \cline{1-7}
375 & 365 & 5 & 554 & 3 & 3 & 2 \\ \cline{1-7}
350 & 390 & 5 & 604 & 3 & 4 & 2 \\ \cline{1-7}
325 & 415 & 5 & 645 & 3 & 4 & 2 \\ \cline{1-7}
300 & 440 & 5 & 690 & 3 & 3 & 2 \\ \cline{1-7}
275 & 465 & 5 & 728 & 3 & 5 & 2 \\ \cline{1-7}
250 & 490 & 5 & 765 & 3 & 3 & 2 \\ \cline{1-7}
225 & 515 & 5 & 801 & 3 & 4 & 2 \\ \cline{1-7}
200 & 540 & 5 & 822 & 3 & 4 & 2 \\ \cline{1-7}
175 & 565 & 5 & 828 & 3 & 3 & 2 \\ \cline{1-7}
150 & 590 & 5 & 828 & 3 & 1 & 1 \\ \cline{1-7}
125 & 615 & 5 & 825 & 3 & 2 & 1 \\ \cline{1-7}
100 & 640 & 5 & 826 & 3 & 3 & 2 \\ \cline{1-7}
75 & 665 & 5 & 813 & 3 & 3 & 2 \\ \cline{1-7}
50 & 690 & 5 & 808 & 3 & 3 & 2 \\ \cline{1-7}
25 & 715 & 5 & 796 & 3 & 1 & 1 \\ \cline{1-7}
0 & 740 & 5 & 788 & 3 & 6 & 2 \\ \cline{1-7}
\end{tabular}
\caption{Ионизационная камера и сцинтилляционный счетчик}
\label{tab:my_label}
\end{table}

\subsubsection*{Ионизационная камера}

\begin{figure}[H]
    \centering
    \includegraphics[width=0.9\textwidth]{images/graph_3.png}
    \caption{График зависимости тока в ионизационной камере от давления.}
    \label{fig:4}
\end{figure}

По графику определим: $P_{\text{э}} = 529\pm14$ Торр. Аналогично предыдущему пункту найдём экстраполированный пробег $R_\text{э}$ и соответствующую энергию.
\[
R_{\text{э}} = \frac{288\text{ K}}{T} \frac{P}{760\text{ Торр}} \frac{10 - 0.5}{2}\text{ см} = 3,25\pm0,08 \text{ см},
\]
где 0.5 см и 10 см — диаметры первого и второго электродов соответственно.

Определим энергию $\alpha$-частиц $E_{\alpha}$.

\[
    R = 0,32 E^{3/2} \Rightarrow E_{\text{э}} = 4,68 \pm 0,63\text{ см}
\]
\subsubsection*{Сцинтилляционный счетчик}

\begin{figure}[H]
    \centering
    \includegraphics[width=0.9\textwidth]{images/graph_2.png}
    \caption{Зависимость числа частиц в зависимости от давления.}
    \label{fig:4}
\end{figure}

Построим график и определим средний и экстраполированный пробег $\alpha$-частиц в воздухе при условиях опыта. Приведем найденное значение к нормальному давлению и температуре --- $P = 760 \text{Торр}, T = 288 \text{К}$. Выразим его в см и г/см$^2$.

$P_\text{ср}=115\pm10$Торр и $P_\text{э} = 204\pm5$Торр будем считать так же как и в предыдущем пункте.

Так как $\alpha$-частицы не могут достигнуть люминофора при обычном давлении, то свободный пробег будет равен расстоянию между препаратом и люминофором — 9 см. Следовательно, мы можем пересчитать средний и экстраполированные свободные пробеги частиц к давлению 760 Торр и температуре 15$^\circ$:
\[
R = \frac{288\text{ K}}{T}\cdot\frac{P}{760\text{ Торр}}\cdot9\text{ см}.
\]

\[
R_\text{ср} = 1,37\pm0,12 \text{ см} \quad R_\text{э} = 2,44\pm0,06 \text{ см}
\]


Определим энергию $\alpha$-частиц.

\[
    R = 0,32 E^{3/2} \Rightarrow E_{\text{э}} = 3,88\pm0,06\text{ МэВ}, \quad E_{\text{ср}} = 2,64\pm0,15\text{ МэВ}
\]

Определим толщину слюдяной пластинки как разность между длинами экстраполированного пробега в Счетчике Гейгера и Сцинтилляционном счетчике:
\[
d_\text{пластинки} = 58 \pm 12 \text{ мм}.
\]


\section*{Вывод}
Таким образом, в ходе работы были получены следующие результаты:
\begin{itemize}
    \item измерены экстраполированный и средний пробеги $\alpha$-частиц тремя различными способами: 
    
        \begin{enumerate}
            \item Счетчик Гейгера: 
        \[
            R_{\text{э}} = 1,86 \pm0,07~ \text{ см} \quad R_{\text{ср}} = 1,73 \pm 0,01 \text{ см}
        \]    
        
            \item Сцинтилляционный счетчик: 
        
        \[
            R_\text{ср} = 1,37\pm0,12 \text{ см} \quad R_\text{э} = 2,44\pm0,06 \text{ см}
        \]
            
            \item Ионизационная камера: 
        
        \[
            R_{\text{э}} = 3,25\pm0,08 \text{ см}
        \]
        
        \end{enumerate}
    
    \item Определена энергия $\alpha$-частицы тремя различными способами:
    
        \begin{enumerate}
            \item Счетчик Гейгера:
        
        \[
            E_{\text{э}} = 3,23\pm0,77~ \text{МэВ},  \quad E_{\text{ср}} = 3,07 \pm 0,15~\text{МэВ}
        \]
        
            \item Сцинтилляционный счетчик:
        
        \[
            E_{\text{э}} = 3,88\pm0,06\text{ МэВ}, \quad E_{\text{ср}} = 2,64\pm0,15\text{ МэВ}
        \]
        
            \item Ионизационная камера:
        
        \[
            E_{\text{э}} = 4,68 \pm 0,63\text{ см}
        \]
        
        \end{enumerate}
    
    
    
    \item Определена толщина слюдяной пластинки: $d_\text{пластинки} = 58 \pm 12 \text{ мм}$.
    
    \item Результаты вычислений энергий для экстраполированных и средних пробегов привели к заниженным значениям. Это обусловлено следующим набором причин:
    	\begin{enumerate}
    		\item
    			Пучки частиц обладают конечными размерами, что приводит к угловой расходимости.
    		\item
    			Источник частиц покрыт слюдяной пленкой, что приводит к замедлению $\alpha$-частиц.
    	\end{enumerate}
    
\end{itemize}

\end{document}