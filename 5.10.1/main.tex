\documentclass[12pt]{article}
\usepackage[top=1cm, bottom=2cm, right=1cm, left=1cm]{geometry}
\usepackage{amsfonts, amssymb, amsmath, hyperref}
\usepackage{graphicx}
\usepackage[T1, T2A]{fontenc}% T2A for Cyrillic font encoding
\usepackage[english, russian]{babel}
\usepackage[justification=centering]{caption}
\usepackage{wrapfig}
\usepackage{placeins}
\usepackage{subcaption}
\usepackage{multirow}
\usepackage{indentfirst}
\usepackage{needspace}
\usepackage{float}
\floatplacement{table}{H}

\begin{document}
\title{\textbf{Лабораторная работа 5.10.1}\\ [2pt]{}}
\date{\today}
\author{Павлов Матвей}

\maketitle

\section*{Аннотация}
Исследовать электронный парамагнитный резонанс в молекуле дифенилпикрилгидразил (ДФПГ), определить $g$-фактор электрона и измерить ширину линий ЭПР.

\section*{Теоретические сведения}

В методе ЭПР изучается резонансное поглощение переменного электромагнитного поля в образце в зависимости от контролируемых экспериментатором внешних условий: постоянного магнитного поля, частоты колебаний переменного поля, температуры и так далее.
 
\begin{figure}[h!]
    \centering
    \includegraphics[width = 0.6\textwidth]{images/1.jpg}
    \caption{Схема резонансного поглощения электромагнитного излучения для изолированного спина $S=1/2$. (a) Зеемановское расщепление спинового уровня в магнитном поле. (б) Переход между подуровнями «снизу-вверх» с поглощением фотона резонансной частоты $h\nu=g\mu B$  . (в) Переход между подуровнями «сверху-вниз» с излучением дополнительного фотона резонансной частоты.}
\end{figure}

Простейшей моделью рассмотрения ЭПР является система из невзаимодействующих частиц со спином $S = 1/2$, помещенная во внешнее магнитное поле. В отсутствии поля проекции спинов на ось совпадают, как и энергии. При подаче магнитного поля и эффекта Зеемана энергии с различными направлениями спина начинают различаться. А если мы в систему направим поток фотонов с этой энергией разности этих энергий, то станут возможны индуцированные переходы межу этими состояниями. 

Для наблюдения этого поглощения необходимо резонансное совпадение частоты излучения с
зеемановским расщеплением спиновых подуровней.

Измеряемой в эксперименте величиной является поглощаемая в образце мощность
излучения. Для увеличения точности измерения желательно увеличить эту поглощаемую
мощность.

Поглощение переменного магнитного поля в образце описывается мнимой частью магнитной
восприимчивости

\[
P_{\text{погл}} = \frac{1}{2}\omega b^2 \chi''(\omega, B)
\]

где $\omega$ --- частота переменного поля, $b$ --- амплитуда однородного по малому образцу переменного поля, $B$ --- постоянное магнитное поле, $\chi''$ ---  мнимая часть высокочастотной магнитной восприимчивости. Магнитная восприимчивость связывает намагниченность $\vec{m}$ с подмагничивающим магнитным полем $\vec{b} : \vec{m} =\chi \vec{b}$. Комплексное представление восприимчивости имеет смысл для описания отклика на переменное поле $\vec{b}=\vec{b}_0 \cdot e^{-i \omega t}$. Тогда $\vec{m} =\chi \vec{b} = (\chi' + i \chi'')\vec{b}_0 e^{-i \omega t}=\chi' \vec{b}_0 e^{-i \omega t} + \chi'' \vec{b}_0 e^{-i \omega t + \frac{pi}{2}}$. Таким образом, действительная часть высокочастотной восприимчивости описывает вклад в
намагниченность, находящийся в фазе с подмагничивающим полем, а мнимая часть — вклад, сдвинутый
относительно подмагничивающего поля по фазе на $\frac{\pi}{2}$. Естественно, при $\omega=0$ мнимая часть восприимчивости $\chi'' = 0$. Сдвиг по фазе отклика (намагниченности) относительно вынуждающей силы
(переменного поля) связан с потерями энергии, поэтому мнимая часть восприимчивости является мерой
диссипации энергии в системе.
\subsection*{Физические причины возникновения резонансного поглощения в
парамагнетике}
Простейшей системой для изучения методом ЭПР является парамагнетик — система слабо
взаимодействующих атомов, ионов или молекул, обладающих собственным магнитным
моментом. Пренебрегая взаимодействием, можно рассмотреть поведение магнитного диполя
в постоянном и переменном магнитном поле. 

В «классическом» подходе рассматривается прецессия магнитного момента во внешнем поле
при отклонении магнитного момента от равновесия. Классический магнитный диполь
стремится выровняться вдоль силовых линий магнитного поля, при отклонении от
равновесия возникает возвращающий механический момент $\vec{T}=\vec{M}\times \vec{B}$ . Так как
магнитный и механический момент иона связаны друг с другом гиромагнитным отношением
$\gamma$ как $\vec{M} =\gamma \vec{I}$ , где $\vec{I}$ - это полный момент импульса, то с учётом уравнения динамики $\frac{d\vec{I}}{dt} = \vec{T}$ получим уравнение прецессии магнитного момента $\frac{d\vec{M}}{dt} = \gamma \vec{M}\times \vec{B}$. Аналогично
с известной задачей о прецессии гироскопа можно заметить, что при отклонении магнитного
момента от направления магнитного поля возникает незатухающая прецессия вокруг
направления поля с угловой скоростью $\vec{\Omega}= - \gamma \vec{B}$ , частота этой прецессии $\Omega_L=\gamma B$
называется ларморовской. При совпадении частоты переменного поля с ларморовской
частотой возможно возникновение резонансного поглощения.

Расщепление терма свободного иона (или молекулы) определяется спектроскопическим
фактором Ланде ($g$-фактором Ланде): $E(m_J)=g\mu_B B m_J$.

\[
g_{\text{эфф}} = \frac{h \nu}{\mu_B B}
\]

\section*{Экспериментальная установка}

В данной работе мы исследуем данную молекулу

\begin{figure}[h!]
    \centering
    \includegraphics[width = 0.4\textwidth]{images/2.jpg}
    \caption{Химическая структура молекулы}
\end{figure}

Схема установки выглядит следующим образом:

\begin{wrapfigure}{r}{0.4\textwidth}
    \centering
    \includegraphics[width = 0.4\textwidth]{images/3.jpg}
  \caption{Схема установки}
\end{wrapfigure}

Схема установки представлена на Рис. 1. Образец (порошок ДФПГ) в стеклянной ампуле помещается внутрь катушки индуктивности, входящей в состав колебательного контура. Входящий в состав контура конденсатор состоит из двух пластин, разделённых воздушным зазором, одна из пластин может перемещаться поворотом штока. Колебания в контуре возбуждаются антенной, соединённой с генератором высокой частоты (ВЧ) Г4-116. Амплитуда колебаний поля в катушке индуктивности измеряется по наводимой в петле связи ЭДС индукции. Высокочастотные колебания ЭДС индукции в приёмном кон-
туре детектируются диодом, измеряемая при помощи осциллографа низкочастотная огибающая этого сигнала пропорциональна квадрату амплитуды колебаний поля в катушке.

Постоянное магнитное поле создаётся пропусканием тока от источника постоянного тока через основные катушки. При этом при помощи вольтметра измеряется падение напряжения на резисторе в цепи основных катушек. Переменное поле небольшой амплитуды создаётся подачей на модуляционные катушки напряжения с регулируемого трансформатора ЛАТР. Для измерения амплитуды колебаний переменного поля используется пробная катушка известной геометрии, подключённая к вольтметру. Пусть поток через неё $\Phi_{\text {проб }},$ тогда ЭДС индукции
$$
\mathcal{E}=-\frac{d \Phi_{\mathrm{npo} 6}}{d t}
$$
Если $I_{\mathrm{ocn}}-$ ток через основную катушку, а $M-$ взаимная индуктивность основной и пробной катушек, то
$$
\Phi_{\mathrm{npo} 6}=M I_{\mathrm{ocn}}
$$
Тогда амплитудное значение ЭДС индукции
$$
\mathcal{E}_{\text {aмп }}=-\frac{d M I_{\text {och }}}{d t}=M \omega I_{\text {aмп }}
$$

Тогда, зная, что
$$
\Phi_{\mathrm{npo} 6}=B_{0} N_{\mathrm{npo} 6} \frac{\pi d_{\mathrm{mpof}}^{2}}{4}=\frac{M U_{R}}{R}=\frac{k U}{\omega}
$$
где $U-$ напряжение на $R$ в резонансе, получим
$$
B_{0}=\frac{4 k U}{\pi \omega d_{\mathrm{npo} 6}^{2} N_{\mathrm{npo} 6}}
$$
Характеристики катушек: пробная катушка $N_{\text {проб }}=46, d_{\text {проб }}=14.6 \pm 0.1$ мм, основная катушка $N_{\text {осн }}=4500, d_{\text {осн }}=0.23 \pm 0.01$ м, модулирующая катушка $N_{\text {мод }}=1500, d_{\text {мод }}=$ $0.30 \pm 0.01$ м.

\section*{Ход работы}

\subsection*{Настройка высокочастотного генератора}

Генератор настраиваем на частоту колебательного контура: в режиме амплитудной модуляции $20 \%$ устанавливаем подстройкой 
частоты генератора b добиваемся максимальной амплитуды сигнала на экране осциллографа. Частоту определяем по шкале генератора, 
$f_{0}=131.1 \pm 0.2$ МГц (погрешность всех измерений частот - цена деления грубой шкалы $\sigma_{f}=0.2 \text{МГц} $). Также для 
вычисления добротности расстроим частоту до достижения сигналом половины от максимального значения. 

\[
f_{+\frac{1}{2}} = 132.0 \pm 0.2 \text{МГц}
\]
\[
f_{-\frac{1}{2}} = 130.4 \pm 0.2 \text{МГц}
\]

Тогда, если $f_{\pm \frac{1}{2}}$ - 
значения частот при достижении половинного сигнала при расстройке генератора в сторону больших и маленьких частот, то добротность
\[
Q = \frac{f_{0}}{f_{+\frac{1}{2}}-f_{-\frac{1}{2}}}
\]
\[
Q = 82 \pm 15
\]
где погрешность рассчитана по формуле
\[
\sigma_{Q} = Q \cdot \sqrt{\left(\frac{\sigma_{f_0}}{f_0}\right)^{2}+\left(\frac{\sigma_{f_{+\frac{1}{2}}}}{f_{+\frac{1}{2}} - f_{-\frac{1}{2}}}\right)^{2}+\left(\frac{\sigma_{f_{-\frac{1}{2}}}}{f_{+\frac{1}{2}} - f_{-\frac{1}{2}}}\right)^{2}}
\]

\subsection*{Наблюдение сигнала резонансного поглощения}

\subsection*{Точная настройка резонансного поля и определение ширины линии}

\subsection*{Калибровка поля электромагнита и определение $g$-фактора}

\section*{Вывод}

Таким образом, в ходе работы были получены следующие результаты:

\begin{itemize}
    \item
\end{itemize}

\end{document}