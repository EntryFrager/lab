\documentclass[12pt]{article}
\usepackage[top=1cm, bottom=2cm, right=1cm, left=1cm]{geometry}
\usepackage{amsfonts, amssymb, amsmath, hyperref}
\usepackage{graphicx}
\usepackage[T1, T2A]{fontenc}% T2A for Cyrillic font encoding
\usepackage[english, russian]{babel}
\usepackage[justification=centering]{caption}
\usepackage{wrapfig}
\usepackage{placeins}
\usepackage{subcaption}
\usepackage{multirow}
\usepackage{indentfirst}
\usepackage{needspace}
\usepackage{float}
\floatplacement{table}{H}

\begin{document}
\title{\textbf{Лабораторная работа 5.10.1}\\ [2pt]{Электро парамагнитный резонанс (ЭПР)}}
\date{\today}
\author{Павлов Матвей}

\maketitle

\section*{Аннотация}
Исследовать электронный парамагнитный резонанс в молекуле дифенилпикрилгидразил (ДФПГ), определить $g$-фактор электрона и измерить ширину линий ЭПР.

\section*{Теоретические сведения}

В методе ЭПР изучается резонансное поглощение переменного электромагнитного поля в образце в зависимости от контролируемых экспериментатором внешних условий: постоянного магнитного поля, частоты колебаний переменного поля, температуры и так далее.
 
\begin{figure}[h!]
    \centering
    \includegraphics[width = 0.6\textwidth]{images/1.jpg}
    \caption{Схема резонансного поглощения электромагнитного излучения для изолированного спина $S=1/2$. (a) Зеемановское расщепление спинового уровня в магнитном поле. (б) Переход между подуровнями «снизу-вверх» с поглощением фотона резонансной частоты $h\nu=g\mu_B$  . (в) Переход между подуровнями «сверху-вниз» с излучением дополнительного фотона резонансной частоты.}
\end{figure}

Простейшей моделью рассмотрения ЭПР является система из невзаимодействующих частиц со спином $S = 1/2$, помещенная во внешнее магнитное поле. В отсутствии поля проекции спинов на ось совпадают, как и энергии. При подаче магнитного поля и эффекта Зеемана энергии с различными направлениями спина начинают различаться. А если мы в систему направим поток фотонов с этой энергией разности этих энергий, то станут возможны индуцированные переходы межу этими состояниями. 

Для наблюдения этого поглощения необходимо резонансное совпадение частоты излучения с
зеемановским расщеплением спиновых подуровней.

Измеряемой в эксперименте величиной является поглощаемая в образце мощность
излучения. Для увеличения точности измерения желательно увеличить эту поглощаемую
мощность.

Поглощение переменного магнитного поля в образце описывается мнимой частью магнитной
восприимчивости

\[
P_{\text{погл}} = \frac{1}{2}\omega b^2 \chi''(\omega, B)
\]

где $\omega$ --- частота переменного поля, $b$ --- амплитуда однородного по малому образцу переменного поля, $B$ --- постоянное магнитное поле, $\chi''$ ---  мнимая часть высокочастотной магнитной восприимчивости. Магнитная восприимчивость связывает намагниченность $\vec{m}$ с подмагничивающим магнитным полем $\vec{b} : \vec{m} =\chi \vec{b}$. Комплексное представление восприимчивости имеет смысл для описания отклика на переменное поле $\vec{b}=\vec{b}_0 \cdot e^{-i \omega t}$. Тогда $\vec{m} =\chi \vec{b} = (\chi' + i \chi'')\vec{b}_0 e^{-i \omega t}=\chi' \vec{b}_0 e^{-i \omega t} + \chi'' \vec{b}_0 e^{-i \omega t + \frac{\pi}{2}}$. Таким образом, действительная часть высокочастотной восприимчивости описывает вклад в
намагниченность, находящийся в фазе с подмагничивающим полем, а мнимая часть — вклад, сдвинутый
относительно подмагничивающего поля по фазе на $\frac{\pi}{2}$. Естественно, при $\omega=0$ мнимая часть восприимчивости $\chi'' = 0$. Сдвиг по фазе отклика (намагниченности) относительно вынуждающей силы
(переменного поля) связан с потерями энергии, поэтому мнимая часть восприимчивости является мерой
диссипации энергии в системе.
\subsection*{Физические причины возникновения резонансного поглощения в
парамагнетике}
Простейшей системой для изучения методом ЭПР является парамагнетик — система слабо
взаимодействующих атомов, ионов или молекул, обладающих собственным магнитным
моментом. Пренебрегая взаимодействием, можно рассмотреть поведение магнитного диполя
в постоянном и переменном магнитном поле. 

В «классическом» подходе рассматривается прецессия магнитного момента во внешнем поле
при отклонении магнитного момента от равновесия. Классический магнитный диполь
стремится выровняться вдоль силовых линий магнитного поля, при отклонении от
равновесия возникает возвращающий механический момент $\vec{T}=\vec{M}\times \vec{B}$ . Так как
магнитный и механический момент иона связаны друг с другом гиромагнитным отношением
$\gamma$ как $\vec{M} =\gamma \vec{I}$ , где $\vec{I}$ - это полный момент импульса, то с учётом уравнения динамики $\frac{d\vec{I}}{dt} = \vec{T}$ получим уравнение прецессии магнитного момента $\frac{d\vec{M}}{dt} = \gamma \vec{M}\times \vec{B}$. Аналогично
с известной задачей о прецессии гироскопа можно заметить, что при отклонении магнитного
момента от направления магнитного поля возникает незатухающая прецессия вокруг
направления поля с угловой скоростью $\vec{\Omega}= - \gamma \vec{B}$ , частота этой прецессии $\Omega_L=\gamma B$
называется ларморовской. При совпадении частоты переменного поля с ларморовской
частотой возможно возникновение резонансного поглощения.

Расщепление терма свободного иона (или молекулы) определяется спектроскопическим
фактором Ланде ($g$-фактором Ланде): $E(m_J)=g\mu_B B m_J$.

\[
g_{\text{эфф}} = \frac{h \nu}{\mu_B B}
\]

\section*{Экспериментальная установка}

В данной работе мы исследуем данную молекулу

\begin{figure}[h!]
    \centering
    \includegraphics[width = 0.4\textwidth]{images/2.jpg}
    \caption{Химическая структура молекулы}
\end{figure}

Схема установки выглядит следующим образом:

\begin{wrapfigure}{r}{0.4\textwidth}
    \centering
    \includegraphics[width = 0.4\textwidth]{images/3.jpg}
  \caption{Схема установки}
\end{wrapfigure}

Схема установки представлена на Рис. 1. Образец (порошок ДФПГ) в стеклянной ампуле помещается внутрь катушки индуктивности, входящей в состав колебательного контура. Входящий в состав контура конденсатор состоит из двух пластин, разделённых воздушным зазором, одна из пластин может перемещаться поворотом штока. Колебания в контуре возбуждаются антенной, соединённой с генератором высокой частоты (ВЧ) Г4-116. Амплитуда колебаний поля в катушке индуктивности измеряется по наводимой в петле связи ЭДС индукции. Высокочастотные колебания ЭДС индукции в приёмном кон-
туре детектируются диодом, измеряемая при помощи осциллографа низкочастотная огибающая этого сигнала пропорциональна квадрату амплитуды колебаний поля в катушке.

Постоянное магнитное поле создаётся пропусканием тока от источника постоянного тока через основные катушки. При этом при помощи вольтметра измеряется падение напряжения на резисторе в цепи основных катушек. Переменное поле небольшой амплитуды создаётся подачей на модуляционные катушки напряжения с регулируемого трансформатора ЛАТР. Для измерения амплитуды колебаний переменного поля используется пробная катушка известной геометрии, подключённая к вольтметру. Пусть поток через неё $\Phi_{\text {проб }},$ тогда ЭДС индукции
$$
U=-\frac{d \Phi_{\mathrm{npo} 6}}{d t}
$$
Если $I_{\mathrm{ocn}}-$ ток через основную катушку, а $M-$ взаимная индуктивность основной и пробной катушек, то
$$
\Phi_{\mathrm{npo} 6}=M I_{\mathrm{ocn}}
$$
Тогда амплитудное значение ЭДС индукции
$$
U_{\text {aмп }}=-\frac{d M I_{\text {och }}}{d t}=M \omega I_{\text {aмп }}
$$

Тогда, зная, что
$$
\Phi_{\mathrm{npo} 6}=B_{0} N_{\mathrm{npo} 6} \frac{\pi d_{\mathrm{mpof}}^{2}}{4}=\frac{M U_{R}}{R}=\frac{k U}{\omega}
$$
где $U-$ напряжение на $R$ в резонансе, получим
$$
B_{0}=\frac{4 k U}{\pi \omega d_{\mathrm{npo} 6}^{2} N_{\mathrm{npo} 6}}
$$
Характеристики катушек: пробная катушка $N_{\text {проб }}=46, d_{\text {проб }}=14.6 \pm 0.1$ мм, основная катушка $N_{\text {осн }}=4500, d_{\text {осн }}=0.23 \pm 0.01$ м, модулирующая катушка $N_{\text {мод }}=1500, d_{\text {мод }}=$ $0.30 \pm 0.01$ м.

\section*{Ход работы}

\subsection*{Настройка высокочастотного генератора}

Генератор настраиваем на частоту колебательного контура: в режиме амплитудной модуляции $20 \%$ устанавливаем подстройкой 
частоты генератора b добиваемся максимальной амплитуды сигнала на экране осциллографа. Частоту определяем по шкале генератора, 
$f_{0}=131.1 \pm 0.2$ МГц (погрешность всех измерений частот - цена деления грубой шкалы $\sigma_{f}=0.2 \text{МГц} $). Также для 
вычисления добротности расстроим частоту до достижения сигналом половины от максимального значения. 

\[
f_{+\frac{1}{2}} = 132.0 \pm 0.2 \text{МГц}
\]
\[
f_{-\frac{1}{2}} = 130.4 \pm 0.2 \text{МГц}
\]

Тогда, если $f_{\pm \frac{1}{2}}$ - 
значения частот при достижении половинного сигнала при расстройке генератора в сторону больших и маленьких частот, то добротность
\[
Q = \frac{f_{0}}{f_{+\frac{1}{2}}-f_{-\frac{1}{2}}} = 82 \pm 15
\]
где погрешность рассчитана по формуле
\[
\sigma_{Q} = Q \cdot \sqrt{\left(\frac{\sigma_{f_0}}{f_0}\right)^{2}+\left(\frac{\sigma_{f_{+\frac{1}{2}}}}{f_{+\frac{1}{2}} - f_{-\frac{1}{2}}}\right)^{2}+\left(\frac{\sigma_{f_{-\frac{1}{2}}}}{f_{+\frac{1}{2}} - f_{-\frac{1}{2}}}\right)^{2}}
\]

\begin{figure}[H]
    \centering
    \includegraphics[width = 0.5\textwidth]{images/3.png}
    \caption{Наблюдение резонанса}
\end{figure}

\subsection*{Наблюдение сигнала резонансного поглощения}

Для наблюдения ЭПР основные катушки электромагнита были подключены к источнику постоянного тока, а модулирующие катушки — к выходу трансформатора. На источнике питания было установлено постоянное выходное напряжение $U_{\text{const}} = 2.2$ В.

Осциллограф был переведен в режим развертки по времени. При плавном увеличении тока через основные катушки на экране наблюдалось возникновение пиков поглощения.

Были зафиксированы два характерных состояния системы:
\begin{enumerate}
    \item \textbf{Начало наблюдения сигнала («Ушки»):} Момент появления краев линии поглощения. Сигнал едва различим, видны характерные всплески на краях развертки.
    \item \textbf{Точный резонанс:} Состояние, при котором пики поглощения на осциллограмме располагаются на равных расстояниях друг от друга.
\end{enumerate}

Эквидистантное (равноудаленное) расположение пиков свидетельствует о том, что постоянное магнитное поле $B_{\text{const}}$ точно совпадает с полем резонанса $B_0$. В этом случае модуляция поля происходит симметрично относительно вершины линии поглощения, и мы пересекаем условие резонанса через равные промежутки времени.

\begin{table}[h!]
    \centering
    \caption{Параметры наблюдения сигнала ЭПР}
    \label{tab:resonance_data}
    \begin{tabular}{|l|c|c|}
        \hline
         & $I$, мА & $U_R$, мВ  \\ \hline
        Появление сигнала («Ушки») & 54 & $40.07$ \\ \hline
        Точный резонанс & 86 & 63.93 \\ \hline
    \end{tabular}
\end{table}

\begin{figure}[H]
    \centering
    \includegraphics[width = 0.4\textwidth]{images/1.png}
    \includegraphics[width = 0.4\textwidth]{images/2.png}
    \caption{Наблюдение ушек. Справа постоянное поле близко к резонансному, пики практически эквидистантны}
\end{figure}

Для дальнейших расчетов $g$-фактора и магнитного поля мы будем использовать точное значение напряжения на шунте в момент резонанса:
\[
    U_{R, \text{рез}} = 63.9 \pm 0.5 \text{ мВ}.
\]
\subsection*{Точная настройка резонансного поля и определение ширины линии}

Для определения ширины линии ЭПР определим по экрану осциллографа полный размах модулирующего поля $A_{\text{полн}}$ и полную ширину кривой резонансного поглощения на полувысоте $A_{\text{1/2}}$.

Не изменяя настроек, внесём пробную катушку внутрь соленоида. Переменное поле наводит в ней ЭДС индукции. Измерения проводились в двух точках (спереди и сзади от образца) для учета неоднородности поля: $U_{\text{перед}} = 4.71$ мВ, $U_{\text{зад}} = 5.90$ мВ.
В качестве расчетного значения примем среднее, а в качестве погрешности — половину разброса значений (так как она превышает инструментальную погрешность вольтметра):
\[ U = 5.31 \pm 0.60~\text{мВ} \quad (\varepsilon_{U} \approx 11\%). \]

Размах и ширина кривой резонансного поглощения определены по осциллограмме (цена деления 1 мВ):
\[ A_{\text{полн}} = 61 \pm 1~\text{мВ}, \quad A_{\text{1/2}} = 9 \pm 1~\text{мВ}. \]

\begin{figure}[H]
    \centering
    \includegraphics[width=0.4\textwidth]{images/4.png} 
    \caption{Осциллограмма линии ЭПР.}
    \label{fig:esr_signal}
\end{figure}

Тогда амплитуда модулирующего поля:
\[B_{\text{мод}} = \dfrac{2\sqrt{2}U}{\pi^2 d_{\text{проб}}^2 N_{\text{проб}}\nu} = 3.10 \pm 0.36~\text{мТл},\]
где использованы параметры пробной катушки $N_{\text{проб}}=46$, $d_{\text{проб}} = 14.6 \pm 0.1$ мм, $\nu = 50$ Гц.
Погрешность $B_{\text{мод}}$ рассчитана по формуле:
\[\sigma_{B_{\text{мод}}}=\sqrt{\left(\dfrac{\partial B_{\text{мод}}}{\partial U} \right)^2 \sigma^2_U + \left(\dfrac{\partial B_{\text{мод}}}{\partial d_{\text{проб}}} \right)^2 \sigma^2_{d_{\text{проб}}}}\]
Полуширину на полувысоте линии резонансного поглощения посчитаем по формуле:
\[\Delta B = \dfrac{A_{1/2}}{A_{\text{полн}}} \cdot B_{\text{мод}} = \frac{9}{61} \cdot 3.10 = 0.46 \pm 0.07~\text{мТл},\]
Погрешность рассчитана по формуле:
\[\sigma_{\Delta B} = \sqrt{ \left(\dfrac{\partial \Delta B}{\partial A_{\text{полн}}} \right)^2 \sigma^2_{A_{\text{полн}}} +  \left(\dfrac{\partial \Delta B}{\partial A_{\text{1/2}}} \right)^2 \sigma^2_{A_{\text{1/2}}} + \left(\dfrac{\partial \Delta B}{\partial B_{\text{мод}}} \right)^2 \sigma^2_{B_{\text{мод}}} }\]

\subsection*{Калибровка поля электромагнита и определение $g$-фактора}

Для определения поля резонансного поглощения найдём связь между падением напряжения на резисторе в цепи основной катушки и магнитным полем. Для этого подадим в основные катушки переменный ток и измеряем при помощи пробной катушки
ЭДС индукции.

Переключим основные катушки на ЛАТР, переведём вольтметр, измеряющий
падение напряжения на резисторе в цепи основных катушек, в режим измерений на
переменном токе, установите ток через катушки, близкий к значению тока при наблюдении резонансного поглощения, измерим в этих условиях ЭДС индукции в пробных катушках. 
Для контроля однородности поля вносим катушку в центр магнита с передней  и задней
стороны установки. Для повышения точности калибровочные измерения проведём при нескольких значениях тока через катушку. 
$U_{\text{сред}} = \left( U_{\text{перед}} + U_{\text{зад}}\right)/2$. Погрешность измерений вольтметра 0.03\% + 4 единицы последнего знака, 
в нашем случае первой частью погрешности можно пренебречь и принять $\sigma_U = 0.04~\text{мВ}$.

\begin{table}[H]
    \centering
    \caption{Данные калибровки магнитного поля }
    \label{tab:calibration}
    \begin{tabular}{|c|c|c|c|}
        \hline
        $U_R$, мВ & $U_{\text{перед}}$, мВ & $U_{\text{зад}}$, мВ & $U_{\text{сред}}$, мВ \\ \hline
        $5.2$ & 0.87 & 0.96 & $0.91 \pm 0.06$ \\
        $15.1$ & 2.61 & 2.89 & $2.75 \pm 0.15$ \\
        $25.1$ & 4.34 & 4.70 & $4.52 \pm 0.18$ \\
        $35.1$ & 6.12 & 6.47 & $6.30 \pm 0.18$ \\
        $45.1$ & 7.87 & 8.32 & $8.10 \pm 0.23$ \\
        $55.4$ & 9.63 & 10.30 & $9.97 \pm 0.34$ \\
        $65.5$ & 11.43 & 12.04 & $11.74 \pm 0.31$ \\
        $76.4$ & 13.40 & 13.91 & $13.66 \pm 0.26$ \\
        $85.4$ & 14.99 & 15.71 & $15.35 \pm 0.36$ \\
        $95.1$ & 16.81 & 17.37 & $17.09 \pm 0.28$ \\ \hline
    \end{tabular}
\end{table}

Построим график зависимости средней ЭДС индукции $U_{\text{сред}}$ от напряжения $U_R$.
Угловой коэффициент наклона прямой $k$ составляет:
\[ k \approx 0.180 \pm 0.004\]

\begin{figure}[H]
    \centering
    \includegraphics[width=0.7\textwidth]{images/graph_1.png}
    \caption{Калибровочный график зависимости ЭДС от напряжения}
    \label{fig:calibration}
\end{figure}

Теперь мы можем посчитать индукцию основного магнитного поля:
\[B_0 = \dfrac{4k U_{R, \text{рез}}}{\pi \omega d_{\text{проб}}^2 N_{\text{проб}}} = 4.75 \pm 0.13~\text{мТл}.\]
Погрешность рассчитывалась по формуле:
\[\sigma_{B_0} = \sqrt{ \left( \dfrac{\partial B_0}{\partial k}\right)^2 \sigma_{k}^2 + \left( \dfrac{\partial B_0}{\partial U_R}\right)^2 \sigma_{U_R}^2 + \left( \dfrac{\partial B_0}{\partial d_{\text{проб}}}\right)^2 \sigma_{d_{\text{проб}}}^2}.\]

Тогда экспериментальное значение $g$-фактора электрона будет равно:
\[g = \dfrac{hf_0}{\mu_B B_0} = 1.97 \pm 0.11.\]
Погрешность считалась по формуле:
\[\sigma_g = \sqrt{ \left( \dfrac{\partial g}{\partial f_0}\right)^2 \sigma_{f_0}^2 + \left( \dfrac{\partial g}{\partial B_0}\right)^2 \sigma_{B_0}^2}.\]

Истинное значение $g$-фактора для ДФПГ составляет $g = 2.0036$. Полученное экспериментальное значение совпадает с теоретическим в пределах погрешности эксперимента.

\subsection*{Измерение на нескольких частотах}

Для проверки линейности зависимости зеемановского расщепления от магнитного поля были проведены измерения положения резонансных пиков на экране осциллографа при различных частотах ВЧ-генератора. Ток через основные катушки поддерживался постоянным.

$A$ — расстояние между пиками резонансного поглощения на экране, $B$ — полная ширина развертки (постоянна и равна 79 делений).

\begin{table}[H]
    \centering
    \caption{Зависимость положения резонанса от частоты}
    \label{tab:freq_dep}
    \begin{tabular}{|c|c|c|}
        \hline
        $\nu$, МГц & $A$, дел. & $A/B$ \\ \hline
        110 & 26 & 0.329 \\
        120 & 30 & 0.380 \\
        140 & 36 & 0.456 \\
        150 & 38 & 0.481 \\
        160 & 41 & 0.519 \\ \hline
    \end{tabular}
\end{table}

Построим график зависимости частоты генератора $\nu$ от расстояния между пиками $A$. Поскольку магнитное поле линейно зависит от смещения на осциллограмме (при фиксированной модуляции), линейный характер полученного графика подтверждает справедливость соотношения $h\nu = g\mu_B B$.

\begin{figure}[H]
    \centering
    \includegraphics[width=0.7\textwidth]{images/graph_2.png}
    \caption{Зависимость резонансной частоты от расстояния между пиками.}
    \label{fig:freq_graph}
\end{figure}

Аппроксимация экспериментальных точек прямой линией $\nu = k \cdot A + b$ подтверждает линейную зависимость энергии расщепления спиновых подуровней от величины магнитного поля.

Для расчета $g$-фактора используем угловой коэффициент наклона $k = 3.39 \pm 0.17$ МГц/дел, полученный из экспериментального графика.

Свяжем изменение частоты $\Delta f$ с изменением расстояния между пиками $\Delta A$.
$\Delta f = \frac{g \mu_B}{h} \Delta B$.

Масштаб развертки по оси X определяется полной шириной развертки $B$ и амплитудой модуляции $B_{\text{мод}}$. Полная ширина экрана соответствует размаху поля $2B_{\text{мод}}$.
\[ \Delta B = \frac{B_{\text{мод}}}{B} \Delta A \]

Подставим это в выражение для частоты и выразим $g$:
\[k = \frac{\Delta f}{\Delta A} = \frac{g \mu_B}{h} \frac{B_{\text{мод}}}{B}\]

\[
g = \frac{h \cdot B}{\mu_B \cdot B_{\text{мод}}} \cdot k \approx 2.06 \pm 0.26
\]

Погрешность считалась по данной формуле:

\[\frac{\sigma_g}{g} = \sqrt{ \left( \frac{\sigma_k}{k_{fA}} \right)^2 + \left( \frac{\sigma_{B_{\text{мод}}}}{B_{\text{мод}}} \right)^2 } \]

\section*{Вывод}

Таким образом, в ходе работы были получены следующие результаты:

\begin{itemize}
    \item Определены значения резонансной частоты и добротности контура:
    \[
        f_0 = 131.1 \pm 0.2 \text{ МГц}
    \]
    \[
        Q = 82 \pm 15
    \]
    \item Определены амплитуда модулирующего поля и полуширина на полувысоте линии резонансного поглощения:
    \[
        B_{\text{мод}} = 3.10 \pm 0.36 \text{ мТл}
    \]
    \[
        \Delta B = 0.46 \pm 0.07 \text{ мТл}
    \]

    Малая ширина линии свидетельствует об обменном сужении, обусловленном взаимодействием спинов в образце.

    \item Посчитан g-фактор электрона:
    \[
        g = 1.97 \pm 0.11
    \]

    \item Посчитан g-фактор путем измерения на нескольких частотах:
    \[
        g = 2.06 \pm 0.26
    \]

    Истинное значение $g$-фактора для ДФПГ составляет $g = 2.0036$. Полученное экспериментальное значение совпадает с теоретическим в пределах погрешности эксперимента.
\end{itemize}

\end{document}