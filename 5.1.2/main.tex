\documentclass[12pt]{article}
\usepackage[top=1cm, bottom=2cm, right=1cm, left=1cm]{geometry}
\usepackage{amsfonts, amssymb, amsmath, hyperref}
\usepackage{graphicx}
\usepackage[T1, T2A]{fontenc}% T2A for Cyrillic font encoding
\usepackage[english, russian]{babel}
\usepackage[justification=centering]{caption}
\usepackage{wrapfig}
\usepackage{placeins}
\usepackage{subcaption}
\usepackage{multirow}
\usepackage{indentfirst}
\usepackage{needspace}
\usepackage{float}
\floatplacement{table}{H}

\begin{document}
\title{\textbf{Лабораторная работа 5.1.2}\\ [2pt]{Исследование эффекта Комптона}}
\date{\today}
\author{Павлов Матвей}

\maketitle

\section*{Аннотация}
С помощью сцинтилляционного спектрометра исследовать энергетический спектр $\gamma$ - квантов, рассеянных на графите.
Определить энергию рассеянных $\gamma$ - квантов в зависимости от угла рассеяния, а также энергию покоя частиц, на которых происходит комптоновское рассеяние.

\section*{Теоретические сведения}

Рассеяние $\gamma$ -лучей в веществе относится к числу явлений, в которых особенно ясно проявляется двойственная природа излучения. 
Волновая теория, хорошо объясняющая рассеяние длинноволнового иЗлучения, испытывает трудности при описании рассеяния рентгеновских и $\gamma$ -лучей. 
Эта теория, в частности, не может объяснить, почему в составе рассеянного излучения, измеренного Комптоном, кроме исходной волны с частотой $\omega_{0}$ появляется дополнительная длинноволновая компонента, 
отсутствующая в спектре первичного излучения.

Появление этой компоненты легко объяснимо, если считать, что $\gamma$-излучение представляет собой поток квантов (фотонов), имеющих энергию $\hbar \omega$ и импульс $p=\hbar \omega / c$. 
Эффект Комптона - увеличение длины волны рассеянного излучения по сравнению с падающим - интерпретируется как результат упругого соударения двух частиц: $\gamma$ -кванта (фотона) и свободного электрона.

Рассмотрим элементарную теорию эффекта Комптона. Пусть электрон до соударения покоился (его энергия равна энергии покоя $m c^{2}$ ), a 
$\gamma$ -квант имел начальную энергию $\hbar \omega_{0}$ и импульс $\hbar \omega_{0} / c .$ После соударения электрон приобретает энергию $\gamma m c^{2}$ 
и импульс $\gamma m v,$ где $\gamma=$ $=\left(1-\beta^{2}\right)^{-1 / 2}, \beta=v / c,$ a $\gamma$ -квант рассеивается на некоторый угол $\theta$ по отношению к первоначальному направлению движения. 
Энергия и импульс $\gamma$ -кванта становятся соответственно равным и $\hbar \omega_{1}$ и $\hbar \omega_{1} / c($ рис. 1$)$. Запишем для рассматриваемого процесса законы сохранения энергии и импульса:

\[
m c^{2}+\hbar \omega_{0}=\gamma m c^{2}+\hbar \omega_{1}
\]
\[
\frac{\hbar \omega_{0}}{c}=\gamma m v \cos \varphi+\frac{\hbar \omega_{1}}{c} \cos \theta
\]
\[
\gamma m v \sin \varphi=\frac{\hbar \omega_{1}}{c} \sin \theta
\]

Решая совместно эти уравнения и переходя от частот $\omega_{0}$ и $\omega_{1}$ к длинам волн $\lambda_{0}$ и $\lambda_{1}$, 
нетрудно получить, что изменение длины волны рассеянного излучения равно

\[
\Delta \lambda=\lambda_{1}-\lambda_{0}=\frac{h}{m c}(1-\cos \theta)=\Lambda_{\mathrm{K}}(1-\cos \theta)
\]

где $\lambda_{0}$ и $\lambda_{1}$ - длины волн $\gamma$ -кванта до и после рассеяния, а величина

\[
\Lambda_{\mathrm{K}}=\frac{h}{m c}=2,42 \cdot 10^{-10} \mathrm{cm}
\]

Основной целью данной работы является проверка соотношения(1). Применительно к условиям нашего опыта формулу
(1) следует преобразовать от длин волн к энергии $\gamma$ -квантов. Как нетрудно показать, соответствующее выражение имеет вид

\[
\frac{1}{\varepsilon(\theta)}-\frac{1}{\varepsilon_{0}}=1-\cos \theta
\]

Здесь $\varepsilon_{0}=E_{0} /\left(m c^{2}\right)-$ выраженная в единицах $m c^{2}$ энергия $\gamma$ -квантов, падающих на рассеиватель, 
$\varepsilon(\theta)$ - выраженная в тех же единицах энергия квантов, испытавших комптоновское рассеяние на угол $\theta, m-$ масса электрона.

\section*{Экспериментальная установка}

Источником излучения 1 служит $137 \mathrm{Cs}$, испускающий $\gamma$-лучи с энергией $662$ кэВ, который помещён в толстостенный свинцовый контейнер с коллиматором. 
Сформированный коллиматором узкий пучок $\gamma$-квантов попадает на графитовую мишень 2 (цилиндр диаметром 40 мм и высотой 100 мм), испытывает рассеяние и регистрируется 
сцинтилляционным счётчиком, состоящим из фотоэлектронного умножителя (ФЭУ) и сцинтиллятора - выходное окно сцинтиллятора находится в оптическом контакте с фотокатодом ФЭУ. Сигналы, возникающием в аноде ФЭУ, 
подаются на компьютер для амплитудного анализа. Кристалл и ФЭУ расположены в светонепроницаемом блоке, укреплённого на горизонтальной штанге, которая может вместе с ним вращаться, 
угол поворота отсчитывается по лимбу 6. 

\begin{figure}[H]
    \centering
    \subfloat{{\includegraphics[width=0.6\textwidth]{images/fig1.jpg}}}
    \subfloat{{\includegraphics[width=0.4\textwidth]{images/2.png}}}
    \caption{Блок-схема установки по изучению рассеяния $\gamma$ - квантов и блок-схема измерительного комплекса}
    \label{fig:scheme}
\end{figure}

Головная часть сцинтилляционного блока закрыта свинцовым коллиматором 5, который формирует входной пучок и защищает детектор от постороннего излучения, 
в основном $\gamma$-квантов, проходящих через стенки защитного контейнера источника. При больших углах измерения для дополнительной защиты между контейнером и источником и детектором ставится свинцовый экран.

\begin{figure}[H]
    \centering
    \includegraphics[width=0.3\textwidth]{images/3.png}
    \caption{Амплитудное распределение импульсов, возникающих в сцинтилляторе}
    \label{fig:graph_1}
\end{figure}

Измерительный комплекс состоит из ФЭУ, питаемого от высоковольтного выпрямителя ВСВ, усилителя-анализатора УА, являющегося входным интерфейсом компьютера ЭВМ, управляемого с клавиатуры КЛ. 
Информация отображается на дисплее Д. 
При работе ФЭУ в спектрометрическом режиме величина выходного электрического импульса пропорциональна энергии регистрируемого $\gamma$-кванта. 
В итоге возникает распределение электрических импульсов (Рис. 2), имеющее фотопик, положение вершины которого нас будет интересовать. 
Левее фотопика начинается непрерывный спектр комптоновских электронов, который сохраняется при любом угле рассеяния. 
Номер канала на распределении соответствует энергии регистрируемой частицы.

\section*{Обработка результатов}

Для обработки результатов предлагается использовать данную формулу. Преобразуем ее, для удобного применения.

\[
mc^2\left(\frac{1}{E(\theta)} - \frac{1}{E(0)}\right) = 1 - \cos \theta
\]

Выполним замену $\frac{E(\theta)}{mc^2} = N(\theta)A$, где $A$ - неизвестный коэффициент пропорциональности между $E(\theta)$ и $N(\theta)$.

\[
\frac{1}{N(\theta)} - \frac{1}{N(0)} = A(1 - \cos \theta)
\]

\[
\frac{1}{N(\theta)} = A(1 - \cos \theta) + \frac{1}{N(0)}
\]

Если представлять в виде $y = Ax + B$, то $y = \frac{1}{N(\theta)}$, $x = (1 - \cos \theta)$, $B = \frac{1}{N(0)}$.

Таким образом, $\frac{E(0)}{mc^2} = BA$, откуда следует, что энергия покоя электрона:

\[
mc^2 = E(0) * \frac{B}{A},
\]

где $E(0) = E_{\gamma} = 662 \text{ кэВ}$ - энергия электронов, рассеянных вперед - равна энергии $\gamma$ - лучей, испускаемых источником (в нашем случае $137 \mathrm{Cs}$).

Для погрешностей будем использовать формулы:

\[
\sigma_{f(x)} = |f'(x)| \sigma_{x},
\]

и

\[
\sigma_{mc^2} = mc^2 * \sqrt{\left(\frac{\sigma_B}{B}\right)^2 + \left(\frac{\sigma_A}{A}\right)^2}
\]

\newpage

\section*{Ход работы}

Устанавливая сцинтилляционный счетчик под разными углами $\theta$ к первоначальному направлению полета $\gamma$ - квантов, 
снимем амплитудные спектры и определим положения фотопиков для каждого значения угла $\theta$. Картина, наблюдаемая на дисплее компьютера, представлена на Рис. 3а)

\begin{table}[H]
    \centering
    \caption{Измеренные величины и их погрешности}
    \begin{tabular}{|c|c|c|c|c|c|c|c|c|c|c|c|c|c|}
    \hline
    $\theta, ^{\circ}$                            & 0     & 10    & 20    & 30    & 40    & 50    & 60    & 70    & 80    & 90    & 100   & 110   & 120   \\ \hline
    $\sigma_{\theta}, ^{\circ}$                   & 3     & 3     & 3     & 3     & 3     & 3     & 3     & 3     & 3     & 3     & 3     & 3     & 3     \\ \hline
    $N(\theta),$ левый полупик                    & 790   & 871   & 757   & 708   & 646   & 570   & 506   & 449   & 411   & 361   & 334   & 304   & 290   \\ \hline
    $N(\theta),$ правый полупик                   & 882   & 959   & 887   & 819   & 784   & 720   & 582   & 516   & 477   & 425   & 386   & 352   & 332   \\ \hline
    $N(\theta),$ главный пик                      & 834   & 918   & 818   & 765   & 722   & 670   & 542   & 477   & 446   & 396   & 361   & 328   & 305   \\ \hline
    $\sigma_{N(\theta)}$                          & 46    & 44    & 65    & 56    & 69    & 75    & 38    & 34    & 33    & 32    & 26    & 24    & 21    \\ \hline
    $1 - \cos \theta$                             & 0     & 0.015 & 0.061 & 0.134 & 0.234 & 0.357 & 0.500 & 0.658 & 0.826 & 1.000 & 1.174 & 1.342 & 1.500 \\ \hline
    $\sigma_{1-\cos \theta} $                     & 0     & 0.009 & 0.018 & 0.026 & 0.034 & 0.040 & 0.045 & 0.049 & 0.052 & 0.052 & 0.052 & 0.049 & 0.045 \\ \hline
    $\frac{1}{N(\theta)} \times 10^{-3}$          & 1.199 & 1.089 & 1.222 & 1.307 & 1.385 & 1.493 & 1.845 & 2.096 & 2.242 & 2.525 & 2.770 & 3.049 & 3.279 \\ \hline
    $\sigma_{\frac{1}{N(\theta)}} \times 10^{-3}$ & 0.066 & 0.052 & 0.097 & 0.095 & 0.132 & 0.167 & 0.129 & 0.147 & 0.166 & 0.204 & 0.200 & 0.223 & 0.226 \\ \hline
    \end{tabular}
    \label{table:power}
\end{table}

\begin{figure}[H]
    \centering
    \subfloat{\includegraphics[width=10cm]{images/4.png}}
    \subfloat{\includegraphics[width=10cm]{images/fig2.png}}
    \caption{а) Картина, наблюдаемая на дисплее; б) Спектры для углов $\theta = 0, 30, 60, 90, 120 ^{\circ} C$}
    \label{fig:subgraph}
\end{figure}

По полученным экспериментальным данным построим график зависимости $\frac{1}{N(\theta)}$ от $1 - \cos \theta$ и определим из него коэффициенты $A$ и $B$.

\[
A = (1.46 \pm 0.06) 10^{-3}
\]
\[
B = (1.08 \pm 0.05) 10^{-3}
\]

Энергия покоя электрона равна:

\[
mc^2 = 489.7 \pm 30.3 \text{ кэВ}
\]

\begin{figure}[H]
    \centering
    \includegraphics[width=\textwidth]{images/graph.png}
    \caption{График зависимости $\frac{1}{N(\theta)}$ от $1 - \cos \theta$}
    \label{fig:graph}
\end{figure}


\section*{Вывод}

Таким образом, в ходе работы были получены следующие результаты:

\begin{itemize}
    \item экспериментально проверена формула Комптона, и что $\gamma$ - кванты испытывают упругое рассеяние на свободных частицах.
    \item Определена энергия покоя частиц, на которых происходило комптоновское рассеяние:
    \[
        mc^2 = 489.7 \pm 30.3 \text{ кэВ}
    \]
    Данное значение совпадает с теоретическим в пределах погрешности:
    \[
        mc^2_{theor} = 511 \text{ кэВ}.
    \]
    
\end{itemize}

\end{document}