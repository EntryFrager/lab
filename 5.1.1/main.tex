\documentclass[12pt]{article}
\usepackage[top=1cm, bottom=2cm, right=1cm, left=1cm]{geometry}
\usepackage{amsfonts, amssymb, amsmath, hyperref}
\usepackage{graphicx}
\usepackage[T1, T2A]{fontenc}% T2A for Cyrillic font encoding
\usepackage[english, russian]{babel}
\usepackage[justification=centering]{caption}
\usepackage{wrapfig}
\usepackage{placeins}
\usepackage{subcaption}
\usepackage{multirow}
\usepackage{indentfirst}
\usepackage{needspace}
\usepackage{float}
\floatplacement{table}{H}

\begin{document}
\title{\textbf{Лабораторная работа 5.1.1}\\ [2pt]{Экспериментальная проверка уравнения Эйнштейна для фотоэффекта и определение постоянной Планка}}
\date{\today}
\author{Павлов Матвей}

\maketitle

\section*{Аннотация}
Исследовать зависимость фототока от величины задерживающего потенциала и частоты падающего излучения. Вычислить величину постоянной Планка.
С помощью сцинтилляционного спектрометра исследовать энергетический спектр $\gamma$ - квантов, рассеянных на графите.
Определить энергию рассеянных $\gamma$ - квантов в зависимости от угла рассеяния, а также энергию покоя частиц, на которых происходит комптоновское рассеяние.

\section*{Теоретические сведения}

Явление фотоэффекта --- испускание электронов веществом под действием электромагнитного излучения --- удовлетворительно объясняется в рамках фотонной теории света. Согласно этой теории, фотон с энергией $\hbar\omega$ передаёт свою энергию электрону в металле. Часть этой энергии тратится на преодоление потенциального барьера на поверхности (работа выхода), остальная часть переходит в кинетическую энергию вылетевшего электрона.

Энергетический баланс процесса описывается уравнением Эйнштейна для фотоэффекта:
\begin{equation}
    \hbar\omega = W + E_{\text{max}},
    \label{eq:einstein_original}
\end{equation}
где $W$ --- работа выхода электрона из катода, $E_{\text{max}}$ --- максимальная кинетическая энергия фотоэлектронов. Реальный энергетический спектр вылетающих электронов является непрерывным и простирается от нуля до $E_{\text{max}}$.

Для измерения кинетической энергии фотоэлектронов используется двухэлектродная система (катод и анод). При подаче на анод задерживающего потенциала ($V < 0$) на анод попадают только электроны, кинетическая энергия которых превышает $e|V|$. При некотором значении $V = -V_0$ (запирающий потенциал) фототок прекращается, так как даже самые быстрые электроны не достигают анода (рис.~1).

\begin{figure}[h!]
    \centering
    \includegraphics[width=0.4\textwidth]{./images/ris1.png}
    \caption{Зависимость фототока от напряжения на аноде.}
    \label{fig:I-V}
\end{figure}

Следует отметить, что даже при напряжениях выше потенциала запирания фототок не обрывается мгновенно. Наличие так называемого \textit{теневого тока} обусловлено электронами, которые вылетают с катода с начальной скоростью, направленной не прямо на анод, но могут достичь его за счёт теплового движения или в результате рассеяния. Этот эффект необходимо учитывать при точных измерениях, экстраполируя линейный участок зависимости $\sqrt{I} = f(V)$ до пересечения с осью напряжений.

Максимальная кинетическая энергия фотоэлектронов связана с запирающим потенциалом соотношением:
\[
E_{\text{max}} = eV_0.
\]
Подстановка этого выражения в уравнение (\ref{eq:einstein_original}) даёт уравнение Эйнштейна в форме, удобной для экспериментальной проверки:
\begin{equation}
    eV_0 = \hbar\omega - W.
    \label{eq:einstein_experimental}
\end{equation}

В случае плоскопараллельной геометрии электродов зависимость фототока от напряжения в области задерживающих потенциалов хорошо аппроксимируется выражением:
\begin{equation}
    \sqrt{I} \propto V_0 - V,
    \label{eq:sqrtI-V}
\end{equation}
то есть квадратный корень из силы фототока линейно зависит от напряжения.

Для экспериментальной проверки уравнения Эйнштейна следует определить запирающие потенциалы $V_0$ для различных частот падающего света $\omega$ и построить зависимость $V_0(\omega)$. Согласно уравнению (\ref{eq:einstein_experimental}), эта зависимость является линейной:
\begin{equation}
    V_0(\omega) = \frac{\hbar}{e}\omega - \frac{W}{e}.
    \label{eq:V0-omega}
\end{equation}

По угловому коэффициенту полученной прямой можно определить постоянную Планка:
\begin{equation}
    \frac{dV_0}{d\omega} = \frac{\hbar}{e}.
    \label{eq:hbar}
\end{equation}
Как следует из уравнения (\ref{eq:V0-omega}), угловой коэффициент не зависит от материала катода, что позволяет определить фундаментальную постоянную $\hbar$.

\section*{Экспериментальная установка}

Схема установки приведена на рисунке 2. Свет от источника $S$ с помощью конденсора фокусируется на входную щель призменного монохроматора УМ-2, выделяющего узкий спектральный интервал, и попадает на катод фотоэлемента Ф-25. Для проведения измерений окуляр монохроматора заменяется на блок фотоэлемента.

\begin{figure}[H]
  \centering
  \begin{minipage}[b]{0.45\textwidth}
    \includegraphics[width=\textwidth]{./images/scheme1.png}
    \caption{Схема экспериментальной установки}
  \end{minipage}
  \hfill
  \begin{minipage}[b]{0.45\textwidth}
    \includegraphics[width=\textwidth]{./images/scheme2.png}
    \caption{Схема монохроматора}
  \end{minipage}
\end{figure}

Основные элементы монохроматора представлены на рисунке 3:
\begin{itemize}
\item Входная щель 1 с микрометрическим винтом 9 для регулировки ширины (0.01--4 мм)
\item Коллиматорный объектив 2 с микрометрическим винтом 8 для точной фокусировки спектральных линий
\item Система призм 3, предназначенная для дисперсии света и поворота лучей на $90^\circ$
\item Поворотный столик 6 с винтом 7 и отсчетным барабаном для сканирования спектра
\item Зрительная труба с объективом 4, окуляром 5 и острием указателя 10
\item Корпус 11, оптическая скамья и пульт управления
\end{itemize}

Измеряемая спектральная линия совмещается с острием указателя, а её положение считывается по барабану.

Фотоэлемент имеет фотокатод из $\text{Na}_2\text{, K, Sb, }\text{Cs}$ и конструктивно объединен с усилителем постоянного тока для измерения малых фототоков. Для снятия вольт-амперных характеристик на фотоэлемент подается запирающее напряжение, измеряемое мультиметром. Фототок, пропорциональный показаниям цифрового вольтметра, регистрируется в зависимости от приложенного напряжения.

\newpage

\section*{Ход работы}

Для начала подготовим установку к работе. Для этого расположим неоновую лампу и линзу на оптическую скамью, настроим их так, чтобы свет попадал на входную щель. Далее откроем входную щель, включим подсветку окуляра и настроимся на чёткое изображение кончика указателя. Далее, для улучшения точности, вращая винт 8, настроим изображение спектра света так, чтобы избежать параллакса света и кончика указателя. Установка подготовлена к работе.

\subsection*{Градуировка монохроматора}
Пользуясь таблицой из методических материалом, проградуируем барабан монохроматора по спектру неоновой лампы. Для этого построим (по известным из таблицы данным и полученной информации об углах из эксперимента) график зависимости длины волны света от угла на барабане.

Погрешность отсчета по шкале барабана: $\sigma_{\text{сист}} = 1$

Посчитаем случайную погрешность измерений угла для линии №22 ($\lambda = 5852$~\AA):
\begin{table}[h!]
    \centering
    \begin{tabular}{|c|c|c|c|c|}
    \hline
    № измерения & 1 & 2 & 3 & 4 \\
    \hline
    $\varphi, \; ^\circ$   & 2158 & 2158 & 2154 & 2156 \\
    \hline
    \end{tabular}
\end{table}

Среднее значение: $\varphi_{\text{ср}} = 2156.5^\circ$

Случайная погрешность: 
\[
\sigma_{\text{сл}} = \sqrt{\frac{\sum_{i=1}^{4}(\varphi_{\text{ср}} - \varphi_{i})^2}{N \cdot (N - 1)}} = 1^\circ
\]

Абсолютная погрешность угла:
\[
\sigma_{\varphi} = \sqrt{\sigma_{\text{сл}}^2 + \sigma_{\text{сист}}^2} \approx 1.4^\circ
\]

\begin{table}[H]
    \centering
    \begin{tabular}{|c|c|c|c|c|c|c|c|c|c|c|c|c|c|c|c|c|c|c|c|c|c|c|c|c|c|c|}
    \hline
    n               & 1 & 2 & 3 & 4 & 5 & 6 & 7 & 8 & 9 & 10 & 11 & 12 & 13 \\ \hline
    $\lambda,$ \AA  & 7032 & 6929 & 6717 & 6678 & 6599 & 6533 & 6507 & 6402 & 6383 & 6334 & 6305 & 6267 & 6217 \\ \hline
    $ \varphi, \; ^\circ $       & 2594 & 2565 & 2500 & 2488 & 2462 & 2440 & 2432 & 2395 & 2387 & 2370 & 2359 & 2343 & 2320 \\ \hline
    n               & 14 & 15 & 16 & 17 & 18 & 19 & 20 & 21 & 22 & 23 & 24 & 25 & \\ \hline
    $\lambda,$ \AA  & 6164 & 6143 & 6096 & 6074 & 6030 & 5976 & 5945 & 5882 & 5852 & 5401 & 5341 & 5331 & \\ \hline
    $ \varphi, \; ^\circ $      & 2298 & 2290 & 2275 & 2264 & 2244 & 2217 & 2203 & 2172 & 2156 & 1896 & 1854 & 1848 & \\ \hline
    		
    \end{tabular}
    \caption{Таблица с длинами волн полос спектра и соответствующими им углами на барабане}
    \label{Таблица с длинами волн полос спектра и соответствующими им углами на барабане}
\end{table}

Для аппроксимации графика зависимости длины волны света от угла будем использовать следующую формулу.

\[
\varphi(\lambda) = \varphi_0 + \frac{c}{\lambda - \lambda_0}, \text{где c - некоторая константа}
\]

\begin{figure}[H]
  \centering
    \includegraphics[width=\textwidth]{./images/graph_1.png}
    \caption{Зависимость угла поворота от длины барабана}
\end{figure}

\subsection*{Исследование зависимости фототока от величины запирающего потенциала}

Установим вместо неоновой лампы электрическую, настроим её на резкое изображение на входной щели, затем установим показания вольтметра близким к нулю при закрытом входе монохроматора. После этого откроем входную щель. 

Измерим зависимость фототока от напряжения. Рассмотрев как изменяется фототок в зависимости от напряжения, построим графики зависимости $\sqrt{I} = f(V)$ для 5 разных частот и для каждой из них найдём значение запирающего потенциала.

\begin{table}[H]
    \centering
    \begin{tabular}{|c|c|c|c|c|c|c|c|c|c|} \hline
    \multicolumn{10}{|c|}{$\lambda$, \AA} \\ \hline
    \multicolumn{2}{|c|}{5852} & \multicolumn{2}{|c|}{6074} & \multicolumn{2}{|c|}{6217} & \multicolumn{2}{|c|}{6402} & \multicolumn{2}{|c|}{6717} \\ \hline
    $I$, A & $V$, В & $I$, A & $V$, В & $I$, A & $V$, В & $I$, A & $V$, В & $I$, A & $V$, В \\ \hline
    0.582 & 7.156 & 0.596 & 7.164 & 0.603 & 7.169 & 0.609 & 7.169 & 0.614 & 7.169 \\ \hline
    0.556 & 5.006 & 0.578 & 5.504 & 0.598 & 6.501 & 0.604 & 6.506 & 0.607 & 6.018 \\ \hline
    0.510 & 2.973 & 0.564 & 4.526 & 0.588 & 5.503 & 0.590 & 5.002 & 0.586 & 4.501 \\ \hline
    0.505 & 2.799 & 0.542 & 3.511 & 0.573 & 4.500 & 0.558 & 3.525 & 0.562 & 3.511 \\ \hline
    0.476 & 2.200 & 0.514 & 2.543 & 0.553 & 3.501 & 0.525 & 2.503 & 0.509 & 2.514 \\ \hline
    0.445 & 1.795 & 0.462 & 1.697 & 0.520 & 2.502 & 0.482 & 1.808 & 0.453 & 1.801 \\ \hline
    0.415 & 1.530 & 0.420 & 1.301 & 0.484 & 1.807 & 0.443 & 1.402 & 0.402 & 1.402 \\ \hline
    0.387 & 1.359 & 0.379 & 1.100 & 0.445 & 1.400 & 0.357 & 1.100 & 0.312 & 1.000 \\ \hline
    0.356 & 1.204 & 0.354 & 1.000 & 0.393 & 1.100 & 0.305 & 0.800 & 0.262 & 0.800 \\ \hline
    0.353 & 1.180 & 0.291 & 0.800 & 0.340 & 0.900 & 0.225 & 0.600 & 0.201 & 0.600 \\ \hline
    0.342 & 1.129 & 0.241 & 0.650 & 0.302 & 0.800 & 0.167 & 0.400 & 0.139 & 0.400 \\ \hline
    0.327 & 1.070 & 0.192 & 0.500 & 0.273 & 0.700 & 0.106 & 0.200 & 0.086 & 0.200 \\ \hline
    0.314 & 1.020 & 0.162 & 0.400 & 0.236 & 0.600 & 0.078 & 0.100 & 0.062 & 0.100 \\ \hline
    0.303 & 0.970 & 0.132 & 0.300 & 0.181 & 0.450 & 0.054 & 0.000 & 0.051 & 0.000 \\ \hline
    0.288 & 0.920 & 0.106 & 0.200 & 0.138 & 0.300 & 0.048 & -0.100 & 0.047 & -0.200   \\ \hline
    0.273 & 0.870 & 0.082 & 0.100 & 0.109 & 0.200 & 0.029 & -0.300 & 0.043 & -0.300   \\ \hline
    0.253 & 0.800 & 0.071 & 0.050 & 0.083 & 0.100 & 0.024 & -0.500 & 0.041 & -0.500   \\ \hline
    0.233 & 0.730 & 0.061 & 0.000 & 0.072 & 0.050 & 0.023 & -0.700 & 0.040 & -0.700   \\ \hline
    0.213 & 0.660 & 0.054 & -0.100 & 0.061 & 0.000 & 0.022 & -1.000 & 0.040 & -1.000  \\ \hline
    0.199 & 0.610 & 0.045 & -0.150 & 0.053 & -0.100 & 0.020 & -2.000 & 0.040 & -2.000 \\ \hline
    0.184 & 0.560 & 0.038 & -0.200 & 0.039 & -0.200 &  &  &  &  \\ \hline
    0.168 & 0.500 & 0.033 & -0.250 & 0.030 & -0.300 &  &  &  &  \\ \hline
    0.156 & 0.450 & 0.028 & -0.300 & 0.024 & -0.450 &  &  &  &  \\ \hline
    0.143 & 0.400 & 0.024 & -0.350 & 0.021 & -0.800 &  &  &  &  \\ \hline
    0.132 & 0.350 & 0.023 & -0.400 & 0.020 & -1.203 &  &  &  &  \\ \hline
    0.119 & 0.300 & 0.019 & -0.500 & 0.018 & -2.017 &  &  &  &  \\ \hline
    0.108 & 0.250 & 0.017 & -0.800 & 0.016 & -4.006 &  &  &  &  \\ \hline
    0.097 & 0.200 & 0.016 & -1.200 & 0.015 & -7.149 &  &  &  &  \\ \hline
    0.086 & 0.150 & 0.012 & -2.506 &  &  &  &  &  &  \\ \hline
    0.076 & 0.100 & 0.012 & -3.445 &  &  &  &  &  &  \\ \hline
    0.066 & 0.050 & 0.011 & -4.250 &  &  &  &  &  &  \\ \hline
    0.058 & 0.008 & 0.011 & -7.146 &  &  &  &  &  &  \\ \hline
    0.052 & -0.100 &  &  &  &  &  &  &  &  \\ \hline
    0.041 & -0.170 &  &  &  &  &  &  &  &  \\ \hline
    0.032 & -0.240 &  &  &  &  &  &  &  &  \\ \hline
    0.023 & -0.340 &  &  &  &  &  &  &  &  \\ \hline
    0.018 & -0.440 &  &  &  &  &  &  &  &  \\ \hline
    0.015 & -0.540 &  &  &  &  &  &  &  &  \\ \hline
    0.012 & -0.660 &  &  &  &  &  &  &  &  \\ \hline
    0.012 & -0.800 &  &  &  &  &  &  &  &  \\ \hline
    0.011 & -1.160 &  &  &  &  &  &  &  &  \\ \hline
    0.010 & -1.505 &  &  &  &  &  &  &  &  \\ \hline
    0.008 & -2.507 &  &  &  &  &  &  &  &  \\ \hline
    \end{tabular}
\end{table}

\begin{figure}[H]
  \centering
    \includegraphics[width=1\textwidth]{./images/line1.png}
\end{figure}

\begin{figure}[H]
  \centering
    \includegraphics[width=\textwidth]{./images/line2.png}
\end{figure}

\begin{figure}[H]
  \centering
    \includegraphics[width=\textwidth]{./images/line3.png}
\end{figure}

\begin{figure}[H]
  \centering
    \includegraphics[width=\textwidth]{./images/line4.png}
\end{figure}

\begin{figure}[H]
  \centering
    \includegraphics[width=\textwidth]{./images/line5.png}
\end{figure}

\begin{table}[H]
	\begin{center}
		\begin{tabular}{|c|c|c|c|c|c|c|c|} \hline
			№ & $ \varphi, \; ^\circ $& $ \lambda, \; \mathring{A} $ &  $ \omega, \; 10^{15} $ с$^{-1}$ & $ a $ & $ b $ & $ V_0 $, В & $ \sigma_{V_0} $, В \\ \hline
			1 & 2156 & 5852 & 3.222 & 0.301 & 0.253 & 0.842 & 0.016 \\ \hline
			2 & 2264 & 6074 & 3.102 & 0.341 & 0.262 & 0.770 & 0.017 \\ \hline
			3 & 2320 & 6217 & 3.030 & 0.358 & 0.263 & 0.734 & 0.022 \\ \hline
			4 & 2395 & 6402 & 2.943 & 0.363 & 0.256 & 0.704 & 0.042 \\ \hline
			5 & 2500 & 6717 & 2.802 & 0.349 & 0.226 & 0.647 & 0.042 \\ \hline
		\end{tabular} 
	\end{center}
	\caption{Результаты измерений для разных длин волн, $ y = ax + b $}
	\label{tab:results}
\end{table}

Теперь построим график зависимости $ V_0(\omega) $.

\begin{figure}[h]
  \centering
    \includegraphics[width=0.95\textwidth]{./images/line6.png}
    \caption{Зависимость запирающего потенциала от частоты света}
\end{figure}

Из графика получаем:

\[
k = \frac{dV_0}{d\omega} = (0.50 \pm 0.04) \cdot 10^{-15} (\text{В} \cdot \text{с})
\]

\[
\dfrac{dV_0}{d\omega} = \dfrac{\hbar}{e} \Rightarrow \hbar = 5.015 \cdot 10^{-16} \cdot 1,602 \cdot 10^{-19} \approx (0.803 \pm 0.065) \cdot 10^{-34} \; \text{Дж} \cdot \text{c} 
\]

В пределах погрешности это согласуется с табличным значением $ \hbar = 1,054 \cdot 10^{-34} \; \text{Дж} \cdot \text{c} $.

Оценим красную границу спектра:

\[
\omega_{\text{к}} = (1.555 \pm 0.283) \cdot 10^{15} \; \text{с}^{-1} \Rightarrow \lambda_{\text{к}} = \frac{2 \pi \text{c}}{\omega_{к}} = 12120 \pm 2210 \; \mathring{A}
\]

Найдем работу выхода 

\[
W = \hbar \omega_{\text{к}} = 0.78 \pm 0.13 \; \text{эВ}
\]

\newpage

\section*{Вывод}

Таким образом, в ходе работы были получены следующие результаты:

\begin{itemize}
    \item Исследовали зависимость фототока от величины задерживающего потенциала и частоты падающего излучения;
    \item Вычислили постоянную Планка:
\[
\hbar \approx (0.803 \pm 0.065) \cdot 10^{-34} \; \text{Дж} \cdot \text{c} 
\]

Она совпадает по порядку величины с табличным значением, но в пределах погрешности значения расходятся:

\[
 \hbar = 1,054 \cdot 10^{-34} \; \text{Дж} \cdot \text{c} 
\]

    \item Вычислили красную границу фотоэффекта:
\[
\omega_{\text{к}} = (1.555 \pm 0.283) \cdot 10^{15} \; \text{с}^{-1} 
\]

\[
\lambda_{\text{к}} = 12120 \pm 2210 \; \mathring{A}
\]

Это означает, что при длинах волн больше $\lambda_{\text{к}}$ фотоэффект невозможен.

    \item Вычислили работу выхода:

\[
W = 0.78 \pm 0.13 \; \text{эВ}
\]

\end{itemize}

\end{document}