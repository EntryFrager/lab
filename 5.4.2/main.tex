\documentclass[12pt]{article}
\usepackage[top=1cm, bottom=2cm, right=1cm, left=1cm]{geometry}
\usepackage{amsfonts, amssymb, amsmath, hyperref}
\usepackage{graphicx}
\usepackage[T1, T2A]{fontenc}% T2A for Cyrillic font encoding
\usepackage[english, russian]{babel}
\usepackage[justification=centering]{caption}
\usepackage{wrapfig}
\usepackage{placeins}
\usepackage{subcaption}
\usepackage{multirow}
\usepackage{indentfirst}
\usepackage{needspace}
\usepackage{float}
\usepackage{longtable}
\floatplacement{table}{H}

\begin{document}
\title{\textbf{Лабораторная работа 5.4.2}\\ [2pt]{}}
\date{\today}
\author{Павлов Матвей}

\maketitle

\section*{Аннотация}

С помощью магнитного спектрометра исследовать энергетический спектр $\beta$-частиц при распаде ядер $^{137}Cs$ и определить их максимальную энергию. Осуществить калибровку спектрометра по энергии электронов внутренней конверсии $^{137}Cs$.

\section*{Теоретические сведения}
	
Бета-распад - самопроизвольное превращение ядер, при котором их массовое число не изменяется, а заряд увеличивается или уменьшается на единицу.
В данной работе:
\[
^A_Z X \to ^{\ \, A}_{Z+1} X + e^- + \widetilde{\nu}
\]

Величина $W(p_e)$ является плотностью вероятности. Распределение электронов по энергии может быть вычислено теоретически. Для разрешенных переходов вероятность $\beta$-распада просто пропорциональна статистическому весу.
\[
    W(p_e)dp_e \propto p_e^2(E_m-E_e)^2 dp_e
\]
Кинетическая энергия электрона и его импульс связаны друг с другом обычной формулой:
\[
    E = \sqrt{(p_{e}c)^2+(m_{e}c^2)^2}-m_{e}c^2
\]
Формула выше приводит к спектру, имеющему вид широкого колокола (рис.~\ref{fig:spectrum}). Кривая плавно отходит от нуля и столь же плавно, по параболе, касается оси абсцисс в области максимального импульса электронов.

\begin{figure}[h!]
    \centering
    \includegraphics[width=0.35\linewidth]{images/spectrum.jpg}
    \captionof{figure}{Блок-схема установки для изучения $\beta$-спектра.}
    \label{fig:spectrum}
\end{figure}


Дочерние ядра, возникающие в результате $\beta$-распада, нередко оказываются возбужденными. Возбужденные ядра отдают свою энергию либо излучая $\gamma$-квант, либо передавая избыток энергии одному из электронов внутренних оболочек атома. Излучаемые в таком процессе электроны имеют строго определенную энергию и называются \textit{конверсионными}.

Конверсия чаще всего происходит на оболочках K и L. Ширина конверсионной линии является чисто аппаратурной -- по ней можно оценить разрешающую силу спектрометра.

\section*{Экспериментальная установка}

Блок-схема установки для изучения $\beta$-спектров изображена на рис. \ref{fig:test1}. Радиоактивный источник $^{137}$Cs помещен внутрь откачанной трубы. Электроны, сфокусированные магнитной линзой, попадают в счетчик. В газоразрядном счетчике они инициируют газовый разряд и тем самым приводят к появлению электрических импульсов на электродах, которые затем регистрируются счетным прибором.

\begin{figure}[h]
\centering
\begin{minipage}{.5\textwidth}
    \centering
    \includegraphics[width=\linewidth]{images/scheme.png}
    \captionof{figure}{Блок-схема установки для изучения $\beta$-спектра.}
    \label{fig:test1}
\end{minipage}%
\begin{minipage}{.5\textwidth}
    \centering
    \includegraphics[width=\linewidth]{images/equip.png}
    \captionof{figure}{Схема $\beta$-спектрометра с короткой магнитной линзой}
    \label{fig:test2}
\end{minipage}
\end{figure}

Энергию $\beta$-частиц определяют с помощью $\beta$-спектрометров (рис.~\ref{fig:test2}). В работе используется магнитный спектрометр с <<короткой линзой>>. Отметим, что в течение всего опыта геометрия прибора остается неизменной, поэтому импульс сфокусированных электронов пропорционален величине тока:
\[
    p_e = kI
\]

Cвязь между числом частиц, регистрируемых установкой, и функцией $W(p_e)$ выражается формулой:
\[
    N(p_e) \propto W(p_e)p_e,
\]
откуда
\[
    \frac{\sqrt{N}}{p_e^{3/2}} \propto E_m - E
\]

\section*{Ход работы}

Проведем предварительное измерение $\beta$-спектра, изменяя ток магнитной линзы через $0.2$ А, а в близи конверсионного пика через $0.05$ A. Данные представлены в таблице ниже. Первая таблица содержит общие данные с шагом $0.2$ А. Вторая - для конверсионного пика с шагом $0.05$ А.

Погрешность при измерении тока: $\sigma_I = 0.02$ А.

Погрешность для N рассчитываем таким образом ($\tau = 100 \text{ c}$):

\[
N = \frac{n}{\tau} \rightarrow n = N \cdot \tau
\]

\[
\sigma_n = \sqrt{n} \rightarrow \sigma_N = \frac{\sqrt{n}}{\tau} = \sqrt{\frac{N}{\tau}}
\]

\newpage

\begin{longtable}{|c|c|c|}
\hline
\textbf{$I$, А} & \textbf{$N$, с$^{-1}$} & \textbf{$\sigma_N$, с$^{-1}$} \\ 
\hline
\endhead

0.00 & 0.480 & 0.069 \\ \hline
0.20 & 0.570 & 0.075 \\ \hline
0.40 & 0.520 & 0.072 \\ \hline
0.60 & 0.560 & 0.075 \\ \hline
0.80 & 0.680 & 0.082 \\ \hline
1.00 & 0.790 & 0.089 \\ \hline
1.20 & 1.030 & 0.101 \\ \hline
1.40 & 1.360 & 0.117 \\ \hline
1.60 & 1.799 & 0.134 \\ \hline
1.80 & 2.439 & 0.156 \\ \hline
2.00 & 2.929 & 0.171 \\ \hline
2.20 & 3.959 & 0.199 \\ \hline
2.40 & 4.399 & 0.210 \\ \hline
2.60 & 4.489 & 0.212 \\ \hline
2.80 & 4.919 & 0.222 \\ \hline
3.00 & 4.189 & 0.205 \\ \hline
3.20 & 3.769 & 0.194 \\ \hline
3.40 & 2.809 & 0.168 \\ \hline
3.60 & 1.570 & 0.125 \\ \hline
3.80 & 1.370 & 0.117 \\ \hline
3.85 & 1.779 & 0.133 \\ \hline
3.90 & 1.999 & 0.141 \\ \hline
3.95 & 2.629 & 0.162 \\ \hline
4.00 & 3.389 & 0.184 \\ \hline
4.05 & 4.299 & 0.207 \\ \hline
4.10 & 5.688 & 0.238 \\ \hline
4.15 & 6.818 & 0.261 \\ \hline
4.20 & 8.618 & 0.294 \\ \hline
4.25 & 8.598 & 0.293 \\ \hline
4.30 & 8.298 & 0.288 \\ \hline
4.35 & 6.958 & 0.264 \\ \hline
4.40 & 6.488 & 0.255 \\ \hline
4.45 & 4.639 & 0.215 \\ \hline
4.50 & 3.769 & 0.194 \\ \hline
4.55 & 2.689 & 0.164 \\ \hline
4.60 & 1.650 & 0.128 \\ \hline
4.65 & 1.180 & 0.109 \\ \hline
4.70 & 0.920 & 0.096 \\ \hline
4.75 & 0.740 & 0.086 \\ \hline
4.80 & 0.670 & 0.082 \\ \hline
5.00 & 0.340 & 0.058 \\ \hline

\end{longtable}

\newpage

Далее построим график зависимости числа отсчетов в фокусирующей катушке. Также вычтем фон из результатов измерений. Для этого построим через две крайние точки ($I = 0$ A и $I = 5$ А) прямую и вычтем ее из спектра.

\begin{figure}[H]
\centering
\begin{minipage}{.5\textwidth}
    \centering
    \includegraphics[width=\linewidth]{images/graph1.png}
    \captionof{figure}{График зависимости $N(I)$ без вычета фона.}
    \label{fig:graph1}
\end{minipage}%
\begin{minipage}{.5\textwidth}
    \centering
    \includegraphics[width=\linewidth]{images/graph2.png}
    \captionof{figure}{График зависимости $N(I)$ с вычетом фона.}
    \label{fig:graph2}
\end{minipage}
\end{figure}

\begin{figure}[H]
    \centering
    \includegraphics[width=0.5\linewidth]{images/pc1.png}
    \captionof{figure}{График зависимости $N(I)$ полученный при обработке данных на компьютере.}
    \label{fig:pc1}
\end{figure}

Построим график зависимости числа отсчетов от импульса электронов. Для этого пересчитаем ток в импульс по такой формуле:

\[
p_e = kI,
\]

где коэффициент $k$ мы определим таким образом ($T_k = 0.624$ МэВ - энергия электронов внутренней конверсии):

Импульс $p$ и кинетическая энергия $T$ связаны релятивистским соотношением:
\[
E^2 = p^2c^2 + m^2c^4
\]

Где $E = T_k + mc^2$. Отсюда выражаем импульс $p$:
\[
p = \sqrt{(T_k + mc^2)^2 - (mc^2)^2} = \sqrt{T_k^2 + 2T_{k}mc^2} \approx 1013.5 \text{ кэВ/c - импульс конверсионных электронов}
\]

Определим значение силы тока при котором у нас наблюдается конверсионный пик и отобразим это на графике.

\begin{figure}[H]
    \centering
    \includegraphics[width=0.6\linewidth]{images/graph3.png}
    \captionof{figure}{Аппроксимация параболой конверсионного пика.}
    \label{fig:graph3}
\end{figure}

То есть, $I_k = 4.26 \pm 0.01$ А. Определим коэффициент $k$ и построим график зависимости $N$ от импульса $p_e$:

\[
1013.5 = k \cdot I_k \rightarrow k = 238 \pm 1 \text{ кэВ/(с$\cdot$А)}
\]

\begin{figure}[H]
\centering
\begin{minipage}{.6\textwidth}
    \centering
    \includegraphics[width=\linewidth]{images/graph4.png}
    \captionof{figure}{График зависимости $N(p)$.}
    \label{fig:graph4}
\end{minipage}%
\begin{minipage}{.4\textwidth}
    \centering
    \includegraphics[width=\linewidth]{images/pc2.png}
    \captionof{figure}{График зависимости $N(p)$ полученный при обработке данных на компьютере.}
    \label{fig:pc2}
\end{minipage}
\end{figure}

Таким образом, мы получаем, что $p_{peak} = 1013 \pm 7 \; \frac{\text{кэВ}}{c}$. Переведем значения импульсов в энергию и построим график зависимости $N(T_e)$. По нему определим энергию конверсионного пика.

\begin{figure}[H]
\centering
\begin{minipage}{.6\textwidth}
    \centering
    \includegraphics[width=\linewidth]{images/graph5.png}
    \captionof{figure}{График зависимости $N(T_e)$.}
    \label{fig:graph4}
\end{minipage}%
\begin{minipage}{.4\textwidth}
    \centering
    \includegraphics[width=\linewidth]{images/pc3.png}
    \captionof{figure}{График зависимости $N(T_e)$ полученный при обработке данных на компьютере.}
    \label{fig:pc2}
\end{minipage}
\end{figure}

\[
T_{peak} = 623.8 \pm 0.6 \text{ кэВ}
\]

Данное значение в пределах погрешности уже совпадает с теоретическим: $T_{theor} = 624 \text{ кэВ}$

Теперь построим график Ферми Кюри по формуле:

\[
mkFermi = \frac{\sqrt{N(p)}}{p^{\frac{3}{2}}} \cdot 10^6
\]

\begin{figure}[H]
\centering
\begin{minipage}{.6\textwidth}
    \centering
    \includegraphics[width=\linewidth]{images/graph6.png}
    \captionof{figure}{График Ферми.}
    \label{fig:graph4}
\end{minipage}%
\begin{minipage}{.4\textwidth}
    \centering
    \includegraphics[width=\linewidth]{images/pc4.png}
    \captionof{figure}{График Ферми полученный при обработке данных на компьютере.}
    \label{fig:pc2}
\end{minipage}
\end{figure}

Получаем, что: $T_{max} = 578 \pm 32 \text{ кэВ}$.

\section*{Вывод}

Таким образом, в ходе работы были получены следующие результаты:

\begin{itemize}
    \item построен и исследован $\beta$-спектр для $^{137}Cs$;
    \item оценена энергия конверсионного пика и максимальная энергия:
    \begin{itemize}
        \item Энергия конверсионного пика: $T_{peak} = 623.8 \pm 0.6 \text{ кэВ}$
        \item Максимальная энергия: $T_{max} = 578 \pm 32 \text{ кэВ}$
    \end{itemize}
\end{itemize}

\end{document}