\documentclass[12pt]{article}
\usepackage[top=1cm, bottom=2cm, right=1cm, left=1cm]{geometry}
\usepackage{amsfonts, amssymb, amsmath, hyperref}
\usepackage{graphicx}
\usepackage[T1, T2A]{fontenc}% T2A for Cyrillic font encoding
\usepackage[english, russian]{babel}
\usepackage[justification=centering]{caption}
\usepackage{wrapfig}
\usepackage{placeins}
\usepackage{subcaption}
\usepackage{multirow}
\usepackage{indentfirst}
\usepackage{needspace}
\usepackage{float}
\floatplacement{table}{H}

\begin{document}
\title{\textbf{Лабораторная работа 5.8.1}\\ [2pt]{Определение постоянных Стефана-Больцмана и Планка из анализа теплового излучения накаленного тела}}
\date{\today}
\author{Павлов Матвей}

\maketitle

\section*{Аннотация}
    При помощи модели абсолютно черного тела (АЧТ) провести измерения температуры оптическим пирометром с исчезающей нитью и термопарой, исследовать излучение накаленных тел с различной испускательной способностью, определить постоянные Планка и Стефана-Больцмана.
	
\section*{Теоретические сведения}

Для измерения температуры разогретых тел, удалённых от наблюдателя, применяют методы оптической пирометрии, основанные на использовании зависимости испускательной способности исследуемого тела от температуры. Различают три температуры, функционально связанные с истинной термодинамической температурой и излучательной способностью тела: радиационную $T_\text{rad}$, цветовую $T_\text{col}$ и яркостную $T_\text{br}$.

В работе измеряется яркостная температура. \textbf{Яркостная температура} - это температура абсолютно чёрного тела, при которой его спектральная испускательная способность равна спектральной испускательной способности исследуемого тела при той же длине волны. Измерение яркостной температуры раскалённого тела производится при помощи оптического пирометра с исчезающей нитью, основанного на визуальном сравнении яркости раскалённой нити с яркостью изображения исследуемого тела. Яркостная температура тела всегда ниже его термодинамической температуры. Это связано с тем, что любое нечёрное тело излучает меньше, чем АЧТ при той же температуре.

\begin{figure}[h]
    \centering
    \includegraphics[width=10cm]{images/fig2.PNG}
    \caption{График зависимости $T = f(T_{br})$ для вольфрам}
    \label{fig:vac}
\end{figure}

По результатам измерений мощности излучения вольфрамовой нити можно судить о справедливости закона Стефана-Больцмана. Если бы нить излучала как АЧТ, то баланс потребляемой и излучаемой энергии определялся бы соотношением 
\begin{equation}\label{black}
    W = \sigma S (T^4 - T_0^4),
\end{equation}
где $W$ - потребляемая нитью электрическая мощность, $S$ - площадь излучающей поверхности нити, $T$ - температура нити, $T_0$ - температура окружающей среды. Однако вольфрамовая нить излучает как серое тел, и излучение её ослаблено по сравнению с АЧТ в $\varepsilon_T$ раз для любой волны при данной температуре тела Т. Тогда предположив, что нить излучает как серое тело и с учётом того, что $T_0 \ll T$, выражение \ref{black} можно переписать в виде
\begin{equation} \label{gray}
    W = \varepsilon_T S \sigma T^4
\end{equation}
В справедливости закона Стефана-Больцмана можно убедиться, построив график зависимости $W(T)$ в логарифмическом масштабе и по углу наклона определить показатель степени $n$ исследуемой температурной зависимости. В пределах погрешности показатель степени должен быть близок к четырём. \par
Также из формулы \ref{gray} можно определить постоянную Стефана-Больцмана. Зная её значение, можно определить значение постоянной Планка по следующей формуле:
\begin{equation}\label{plank}
    h = \sqrt[3]{\frac{2\pi^5k_\text{Б}^4}{15c^2\sigma}}.
\end{equation}

\section*{Экспериментальная установка}

\begin{figure}[H]
    \centering
    \includegraphics[width=0.6\textwidth]{images/scheme8_1.png}
    \caption{Схема установки}
    \label{fig:scheme}
\end{figure}
Экспериментальная установка состоит из оптического пирометра, модели АЧТ, образцов с общим блоком питания, а также вольтметра и амперметра для измерения тока вольфрамовой нити. Оптический пирометр с исчезающей нитью представляет собой зрительную трубу, в плоскости изображения которой находится накаливающаяся проволока. Глядя через встроенный монохроматор предполагается установить температуру проволоки такой, чтобы ее свет был <<таким же>> как свет тела, т.е. их яркостная испускательная способность сравнялась, и нить исчезла на фоне объекта. Так как прибор проградуирован по АЧТ, на экране пирометра будет отображаться яркостная температура объекта.

В качестве модели АЧТ используется керамическая трубка, нагреваемая нихромовой проволокой. Ее температура измеряется хромель-алюминиевой термопарой, подключенной к вольтметру, и позволяющей определить разность комнатной температуры с температурой АЧТ.

В качестве одного из образцов используется вольфрамовая нить лампы накаливания. Напряжение и сила тока через нее измеряются непосредственно соответствующими приборами, что даёт возможность определить ее потребляемую и излучаемую мощность. Другой образец~--- керамическая трубка с кольцами из различных материалов, нагреваемая до сравнительно однородной температуры единой нихромовой нитью. Последним изучаемым образцом является неоновая лампа.

\section*{Ход работы}

\subsection*{Изучение работы оптического пирометра}

С помощью пирометра измерим температуру модели АЧТ и проведем сравнение ее значения со значением температуры, измеренной при помощи термопарного термометра.

\begin{table}[h]
    \centering
    \begin{center}
        \caption{Сравнение температуры нити пирометра и температуры АЧТ}
    \end{center}
    \begin{tabular}{|l|l|l|l|l|l|l|}
    \hline
    $T_{term}$, $C^\circ$          & 1167  & 1159 & 1165 & 1161 & Mean $T$, $C^\circ$& $1163\pm3$  \\ \cline{1-5} \cline{7-7} \hline
    $V$, мВ                & \multicolumn{4}{l|}{44.75}  & $T$, $C^\circ$           & $1091$        \\ \hline
    \end{tabular}
    \label{table:comparison}
\end{table}

Разница между значениями температур измеренных пирометром и термопарой составляет $\approx 5.3 \%$.

\subsection*{Измерение яркостной температуры накаленных тел}

Проверим, что различные тела, накаленные до одинаковой термодинамической температуры, имеют различную яркостную температуру. Для этого нагреем керамическую трубку с образцами до красного каления и измерим яркостную температуру поверхности трубки и колец.

\begin{itemize}
    \item Температура поверхности трубки $\approx 779\pm5\ C^\circ$;
    \item Температура металлических колец $\approx 760\pm5\ C^\circ$;
    \item Температура неметаллических колец $< 700\ C^\circ$ (не определяется пирометром).
\end{itemize}

Несовпадение яркостной температуры у различных тел, имеющих одинаковую термодинамическую температуру, возникает из-за того, что эти две величины связаны, в том числе, через спектральный коэффициент поглощения, который у разных материалов различный.    

\subsection*{Проверка закона Стефана-Больцмана}

В диапазоне 900-2100 $^\circ$C измерим яркостную температуру нити лампы накаливания, а также значение силы тока и напряжения на ней. Также для каждого значения измеренной яркостной температуры найдем термодинамическую температуру вольфрамовой нити лампы, используя \ref{fig:vac}.

\begin{table}[h]
    \centering
    \caption{Зависимость мощности, выделяемой на лампе, от температуры нити накала}
    \begin{tabular}{|c|c|c|c|c|c|c|}
    \hline
    № & $T_{light}$, K & $T$, K ($\sigma_{T}$ = 5 K) & $I \pm \sigma_I$, А & $V \pm \sigma_V$, В & $W \pm \sigma_{W}$, мВт & $ \frac{W}{\varepsilon_T}$, мВт\\ \hline
    1  & 1143 & 1158  & 0.504 $\pm$ 0.003 & 1.582 $\pm$ 0.040 & 797.3  $\pm$ 20.7 & 6272.0  \\ \hline
    2  & 1243 & 1258  & 0.546 $\pm$ 0.004 & 1.973 $\pm$ 0.040 & 1077.3 $\pm$ 23.1 & 7729.2  \\ \hline
    3  & 1343 & 1368  & 0.581 $\pm$ 0.004 & 2.309 $\pm$ 0.040 & 1341.5 $\pm$ 24.9 & 8512.1  \\ \hline
    4  & 1443 & 1488  & 0.625 $\pm$ 0.004 & 2.768 $\pm$ 0.040 & 1730.0 $\pm$ 28.2 & 9763.0  \\ \hline
    5  & 1543 & 1588  & 0.665 $\pm$ 0.004 & 3.192 $\pm$ 0.040 & 2122.7 $\pm$ 31.0 & 10993.9 \\ \hline
    6  & 1643 & 1693  & 0.703 $\pm$ 0.005 & 3.615 $\pm$ 0.040 & 2541.3 $\pm$ 35.6 & 12216.6 \\ \hline
    7  & 1743 & 1793  & 0.754 $\pm$ 0.005 & 4.220 $\pm$ 0.040 & 3182.0 $\pm$ 37.9 & 14332.0 \\ \hline
    8  & 1843 & 1903  & 0.818 $\pm$ 0.005 & 5.002 $\pm$ 0.040 & 4091.6 $\pm$ 41.0 & 17308.7 \\ \hline
    9  & 1943 & 2018  & 0.850 $\pm$ 0.005 & 5.417 $\pm$ 0.040 & 4604.4 $\pm$ 46.3 & 18491.6 \\ \hline
    10 & 2043 & 2123  & 0.892 $\pm$ 0.006 & 5.979 $\pm$ 0.040 & 5333.3 $\pm$ 57.3 & 21418.9 \\ \hline
    11 & 2143 & 2233  & 0.953 $\pm$ 0.006 & 6.825 $\pm$ 0.040 & 6504.2 $\pm$ 60.0 & 26121.3 \\ \hline
    12 & 2243 & 2338  & 0.997 $\pm$ 0.006 & 7.477 $\pm$ 0.040 & 7454.6 $\pm$ 60.0 & 29938.2 \\ \hline
    13 & 2343 & 2443  & 1.042 $\pm$ 0.006 & 8.162 $\pm$ 0.040 & 8504.8 $\pm$ 64.3 & 34155.8 \\ \hline
    \end{tabular}
    \label{table:power}
\end{table}

\begin{figure}[H]
    \centering
    \includegraphics[width=\textwidth]{images/graph_1.png}
    \caption{Зависимость мощности лампы от её термодинамической температуры, логарифмический масштаб}
    \label{fig:graph}
\end{figure}

Представим зависимость $W=f(T)$ в логарифмическом масштабе как $\ln(W) = \ln(\varepsilon_T \sigma S) + n \ln(T)$. 
По углу наклона графика можно определить показатель степени температуры в законе Стефана-Больцмана. 
Его экспериментальное значение равно $n = 3.17 \pm 0.15$, где коэффициент наклона и ошибка измерения получены по методу наименьших квадратов. 
Теоретическое значение $n = 4$ не совпадает с экспериментальным. Это обусловлено тем, что вольфрамовая нить не АЧТ.  

Найдем величину постоянной Стефана-Больцмана для каждого измеренного значения T, превышающего 1700 K, по формуле
\begin{equation*}
    \sigma = \frac{W}{\varepsilon_T S T^4} 
\end{equation*}
Результаты представлены в таблице 2.

\begin{table}[h]
    \centering
    \caption{Зависимость постоянной Стефана-Больцмана от термодинамической температуры}
    \label{bolts}
    \begin{tabular}{|c|c|c|c|c|c|c|c|} \hline
        $T$, К & 1793 & 1903 & 2018 & 2123 & 2233 & 2338 & 2443\\ \hline
        $\varepsilon_T$ & 0,222 & 0,236 & 0,251 & 0,265 & 0,279 & 0.293 & 0.307 \\ \hline
        $\sigma, 10^{-8}$Вт/(м$^2 \cdot$ K$^4$) & 3.85 & 3.68 & 3.07 & 2.75 & 2.60 & 2.37 & 2.16 \\ \hline
        $\sigma_\sigma, 10^{-8}$Вт/(м$^2 \cdot$ K$^4$) & 0.13 & 0.11 & 0.09 & 0.08 & 0.07 & 0.06 & 0.06 \\ \hline
    \end{tabular}
\end{table}   

Усреднением получим значение: $\sigma = (2.93 \pm 0,03) \cdot 10^{-8}$ Вт/(м$^2 \cdot$ K$^4)$. 
Здесь погрешность складывается из среднеквадратичного отклонения при усреднении и измерительной погрешности. 
Полученное значение не совпадает с теоретическим:
    $\sigma_{th} = 5.67\cdot 10^{-8}$ Вт/(м$^2 \cdot$ K$^4)$.

Относительное расхождение составляет $\approx 48.3\%$. 

Оценим значение постоянной Планка:
\begin{equation*}
    h = \sqrt[3]{\frac{2 \pi^5 k_\text{Б}^4}{15 c^2 \sigma}} = (8.25 \pm 0,27) \cdot 10^{-34}\text{ Дж}\cdot\text{с}.
\end{equation*}

Относительное расхождение составляет $\approx 24.5\%$. 

\subsection*{Измерение яркостной температуры неоновой лампочки}

Термодинамическая температура неоновой лампочки примерно равна комнатной, что не соответствует её яркостной температуре ($\approx 829\pm5\ C^\circ$). Объясним почему так происходит:

Настоящая температура стеклянной колбы — это термодинамическая температура, связанная со средней энергией теплового движения атомов и молекул вещества. 
Стекло нагревается только от столкновений быстрых частиц газа со стенками. 
Мощность разряда мала, и при этом стекло хорошо отдаёт тепло в воздух. 

Когда обычное тело нагревается, оно излучает непрерывный спектр: энергия распределена по всем длинам волн.
А в неоновой лампе излучает газовый разряд, и энергия выделяется только на строго определённых длинах волн(590-740 нм), 
соответствующих переходам электронов между уровнями энергии атомов неона (линейчатый спектр). 

Пирометр ловит именно это излучение неона.
Но эта температура не отражает настоящий нагрев стекла. 
Поэтому яркостная температура оказывается высокой, а термодинамическая температура стеклянной колбы остаётся близкой к комнатной.

\newpage

\section*{Вывод}
Таким образом, в ходе работы были получены следующие результаты:
\begin{itemize}
    \item Проведено ознакомление с принципом работы оптического пирометра. В ходе его настройки и работы с моделью АЧТ выяснилось, что разность показаний пирометра и действительной температурой составляет до 6.6\%.
    \item Выяснено, что для различных материалов с одинаковой термодинамической температурой их яркостная температура может не совпадать. Это связано с различием коэффициента спектрального поглощения этих материалов.
    \item Проверена справедливость закона Стефана-Больцмана ($W \propto T^4$). Данную зависимость получить не удалось. Значение степени $T$, определенной в работе, составляло $3.17$. Также по результатам измерения была оценена постоянная Стефана-Больцмана и постоянная Планка.
        \begin{center}
            $\sigma = (2.93 \pm 0.03) \cdot 10^{-8}$ Вт/(м$^2 \cdot$ K$^4$) \\
            $\sigma_{th} = 5.67\cdot 10^{-8}$ Вт/(м$^2 \cdot$ K$^4$) \\
            $ h = (8.25 \pm 0,27) \cdot 10^{-34}\text{ Дж}\cdot\text{с}. $
        \end{center}
    \item С помощью пирометра была определена яркостная температура неоновой лампочки, не являющейся моделью АЧТ. Эта яркостная температура не совпадает с термодинамической.
\end{itemize}

\end{document}