\documentclass[12pt]{article}
\usepackage[top=1cm, bottom=2cm, right=1cm, left=1cm]{geometry}
\usepackage{amsfonts, amssymb, amsmath, hyperref}
\usepackage{graphicx}
\usepackage[T1, T2A]{fontenc}% T2A for Cyrillic font encoding
\usepackage[english, russian]{babel}
\usepackage[justification=centering]{caption}
\usepackage{wrapfig}
\usepackage{placeins}
\usepackage{subcaption}
\usepackage{multirow}
\usepackage{indentfirst}
\usepackage{needspace}
\usepackage{float}
\floatplacement{table}{H}

\begin{document}
\title{\textbf{Лабораторная работа 5.2.2-5.2.3}\\ [2pt]{Изучение спектров атома водорода и молекулы йода}}
\date{\today}
\author{Павлов Матвей}

\maketitle

\section*{Аннотация}
Исследовать спектральные закономерности в оптическом спектре водорода, и спектр поглощения паров йода в видимой области. Вычислить постоянную Ридберга, энергию колебательного кванта молекулы йода и энергию ее диссоциации в основном и возбужденном состояниях.

\section*{Теоретические сведения}

\subsection*{Спектр атомов водорода.}
	
Объяснение структуры спектра излучения атомов требует решения задачи о движении электрона в эффективном поле атома.	Для атома водорода и водородоподобных (одноэлектронных) атомов 	определение энергетических уровней значительно упрощается, так как	квантово-механическая задача об относительном движении электрона (заряд $ -e $, масса $ m_e $) и ядра (заряд $ Z_e $, масса $ M $) сводится к задаче о движении частицы с эффективной массой $ \mu = m_e M /(m_e+M) $ в кулоновском поле $ - Z \epsilon^2 / n $. Длины волн спектральных линий водородоподобного атома описываются формулой
\[
    \dfrac{1}{\lambda_{m n}} = R Z^2 (\dfrac{1}{n^2}-\dfrac{1}{m^2}),
\]
где $ m, \;n \in \mathbb{Z} $, а $ R $ -- постоянная Ридберга.
Эта формула позволяет по энергиям перехода судить о расположении энергетических уровней атома водорода. На рис. \ref{fig:1} изображены энергетические уровни и соответствующие им переходы, определяющие спектр.

\begin{figure}[h!]
    \centering
    \includegraphics[width=0.4\textwidth]{images/1.png}
    \caption{Энергетические уровни атома водорода}
    \label{fig:1}
\end{figure}

В данной работе изучается серия Бальмера, линии которой лежат в видимой области. Для серии Бальмера $ n = 2 $. Величина $ m $ для первых четырех линий этой серии принимает значение 3, 4, 5, 6. Эти линии	обозначаются символами $ H_\alpha,\;H_\beta,\;H_\gamma,\;H_\delta $.

Оценим энергии основного и возбужденного состояний водородоподобного атома. Чтобы найти основное состояние квантовой системы, надо минимизировать, с учетом соотношения неопределенностей, полную энергию. Потенциальная энергия электрона равна кулоновской энергии электрона в поле ядра с зарядом $ Z e $. Так как электрон локализован в области размером $ r $, то его импульс $ p \simeq \hbar / r $, и полная энергия определяется выражением
\[
    E = \dfrac{-Z e^2}{r}+\dfrac{\hbar^2}{2 m_e r^2}.
\]
Приняв за нуль производную этого выражения, получим
\[
    r_\text{Б} = \dfrac{\hbar^2}{Z m_e e^2}.
\]
Это значение радиуса первой орбиты для электрона в поле ядра с зарядом $ Z $ -- боровского радиуса. Подставляя в формулу с полной энергией это значение, получим
\[
    E = -R Z^2,
\]
\[
    R = \dfrac{m_e e^4}{2 \hbar^2}.
\]
Для возбуждённых состояний значения энергий можно найти аналогично, приняв во внимание, что $ p \simeq n \hbar / r $ из условия, что на длине орбиты укладывается целое число волн де Бройля. Отсюда энергия $ n $-го уровня равна 
\[
    E = \dfrac{-R Z^2}{n^2}.
\]

\subsection*{Спектр молекул йода.}

Массы ядер атомов велики по сравнению с массой электрона. Благодаря такой разнице в массах, скорости движения ядер в молекуле малы по сравнению со скоростями электронов. Это даёт возможность	рассматривать электронное движение при неподвижных ядрах, расположенных на определенных расстояниях друг от друга. Определяя уровни энергии такой системы, мы найдем электронные термы молекул. Любой атом в молекуле находится в электрическом поле остальных ее атомов. Оно вызывает расщепление электронных уровней атомов в молекуле. Следует отметить, что при соединении атомов в молекулу заполненные оболочки атомов мало меняются. Существенно может измениться распределение электронной плотности в не до конца заполненных оболочках.

\begin{figure}[H]
    \centering
    \includegraphics[width=0.4\textwidth]{images/3.png}
    \caption{Спектральная картина йода}
    \label{fig:3}
\end{figure}

Спектр молекулярного йода представлен на рис. \ref{fig:3}.
Для расчёта спектра поглощения йода необходимо учесть энергии колебательного и вращательного движения молекул. Видимый спектр состоит из 0-й и 1-й серий Деландра. 2-я серия в 10 раз менее интенсивная, чем 0-я, и поэтому ей пренебрегаем. 

Энергетическое	положение линий поглощения описывается выражением
\[
    h \nu_{0 n_2} = (E_2 - E_1 )+ h \nu_2 \left(n_2+\dfrac{1}{2}\right) - \dfrac{1}{2}h \nu_1.
\]

\section*{Экспериментальная установка}

\begin{figure}[h!]
    \centering
    \includegraphics[width=0.6\textwidth]{images/2.png}
    \caption{Схема экспериментальной установки}
    \label{fig:2}
\end{figure}
Для измерения длин волн спектральных линий используется стеклянно-призменный монохроматор-спектрометр УМ-2, работающий в диапазоне 0{,}38--1{,}00~мкм.

Схема прибора приведена на рис.~\ref{fig:2}. Излучение от источника проходит через регулируемую входную щель~1 и, будучи коллимированным объективом~2, направляется на составную призму~3. После дисперсии свет собирается объективом~4 и наблюдается через окуляр~5. Поворот призмы барабаном со шкалой обеспечивает выбор длины волны. Подсветка указателя~6 позволяет фиксировать положение спектральных линий.

Конструкция прибора позволяет регулировать ширину щели и фокусировку для достижения требуемого спектрального разрешения.

В качестве излучателей использовались:
\begin{itemize}
    \item Неоновая лампа, далее Ne,
    \item Ртутная лампа, далее Hg,
    \item Водородная лампа, далее H,
    \item Лампа накаливания широкого (полного) спектра, свет которой направлялся в кювету с парами йода ДЛЯ исследования спектра его поглощения, далее I.
\end{itemize}

\section*{Ход работы}

\subsection*{Градуировка монохроматора}
Пользуясь таблицой из методических материалов, проградуируем барабан монохроматора по спектру неоновой и ртутной лампы. Для этого построим (по известным из таблицы данным и полученной информации об углах из эксперимента) график зависимости длины волны света от угла на барабане.

Погрешность отсчета по шкале барабана: $\sigma_{\text{сист}} = 1$

Посчитаем случайную погрешность измерений угла для линии №22 ($\lambda = 5852$~\AA):
\begin{table}[h!]
    \centering
    \begin{tabular}{|c|c|c|c|c|}
    \hline
    № измерения & 1 & 2 & 3 & 4 \\
    \hline
    $\varphi, \; ^\circ$   & 2137 & 2136 & 2136 & 2138   \\
    \hline
    \end{tabular}
\end{table}

Среднее значение: $\varphi_{\text{ср}} = 2137^\circ$

Случайная погрешность: 
\[
\sigma_{\text{сл}} = \sqrt{\frac{\sum_{i=1}^{4}(\varphi_{\text{ср}} - \varphi_{i})^2}{N \cdot (N - 1)}} = 1^\circ
\]

Абсолютная погрешность угла:
\[
\sigma_{\varphi} = \sqrt{\sigma_{\text{сл}}^2 + \sigma_{\text{сист}}^2} \approx 1.4^\circ
\]

\begin{table}[H]
    \centering
    \begin{tabular}{|c|c|c|c|c|c|c|c|c|c|c|c|c|c|c|c|c|c|c|c|c|c|c|c|c|c|c|}
    \hline
    n                            & 1 & 2 & 3 & 4 & 5 & 6 & 7 & 8 & 9 & 10 & 11 & 12 & 13 \\ \hline
    $\lambda,$ \AA               & 7032 & 6929 & 6717 & 6678 & 6599 & 6533 & 6507 & 6402 & 6383 & 6334 & 6305 & 6267 & 6217 \\ \hline
    $ \varphi, \; ^\circ $       & 2575 & 2546 & 2480 & 2468 & 2442 & 2420 & 2411 & 2374 & 2364 & 2348 & 2335 & 2320 & 2301 \\ \hline
    n                            & 14 & 15 & 16 & 17 & 18 & 19 & 20 & 21 & 22 & 23 & 24 & 25 & \\ \hline
    $\lambda,$ \AA               & 6164 & 6143 & 6096 & 6074 & 6030 & 5976 & 5945 & 5882 & 5852 & 5401 & 5341 & 5331 & \\ \hline
    $ \varphi, \; ^\circ $       & 2278 & 2270 & 2250 & 2240 & 2220 & 2195 & 2182 & 2151 & 2136 & 1878 & 1837 & 1828 & \\ \hline
    		
    \end{tabular}
    \caption{Таблица с длинами волн полос неонового спектра и соответствующими им углами на барабане}
    \label{neon}
\end{table}

\begin{table}[H]
    \centering
    \begin{tabular}{|c|c|c|c|c|c|c|c|c|c|c|c|c|c|c|c|c|c|c|c|c|c|}
    \hline
    n                            & 1 & 2 & 3 & 4 & 5 & 6 & 7 & 8 \\ \hline
    $\lambda,$ \AA               & 6907 & 6234 & 5791 & 5770 & 5461 & 4916 & 4358 & 4047 \\ \hline
    $ \varphi, \; ^\circ $       & 2539 & 2307 & 2105 & 2094 & 1916 & 1497 & 835 & 280 \\ \hline	
    \end{tabular}
    \caption{Таблица с длинами волн полос ртутного спектра и соответствующими им углами на барабане}
    \label{rtuti}
\end{table}

Для аппроксимации графика зависимости длины волны света от угла будем использовать следующую формулу.

\[
\varphi(\lambda) = \varphi_0 + \frac{c}{\lambda - \lambda_0}, \text{где c - некоторая константа}
\]

\begin{figure}[H]
  \centering
    \includegraphics[width=\textwidth]{./images/graph_1.png}
    \caption{Зависимость угла поворота от длины барабана}
\end{figure}

Полученная зависимость:

\[
\varphi_0 = 3883 \pm 3 ^\circ
\]
\[
c = -6128357 \pm 14561
\]
\[
\lambda_0 = 2346 \pm 3 \text{\AA}
\]

\subsection*{Исследование спектра водорода}

Определим длины волн $H_\alpha,\;H_\beta,\;H_\gamma,\;H_\delta $, с помощью калибровочной формулы. Для оценки ошибки измерения угол поворота барабана для каждой спектральной линии снимался 4 раза

\begin{table}[h]
    \centering
    \begin{tabular}{|l|l|l|l|l|}
        \hline
        & $\varphi_1, \;^\circ$ & $\varphi_2, \;^\circ$ & $\varphi_3, \;^\circ$ & $\varphi_4, \;^\circ$ \\ \hline
        $ H_\alpha$ & $2430\pm 2$      & $2430 \pm 2$ & $2429 \pm 2$ & $2430 \pm 2$       \\ \hline
        $H_\beta$   & $1446\pm 2$      & $1445 \pm 2$ & $1446 \pm 2$ & $1444 \pm 2$        \\ \hline
        $H_\gamma$  & $806\pm 2$       & $810 \pm 2$  & $807 \pm 2$ & $805 \pm 2$     \\ \hline
        $H_\delta$  & $395\pm 2$       & $395\pm 2$   & $396 \pm 2$  & $400 \pm 2$      \\ \hline
    \end{tabular}
    \caption{Экспериментальные значения для спектральных линий водорода}
    \label{tab:H}
\end{table}

Далее, усредним их ($N = 4$) и посчитаем ошибку усреднения по таким формулам:

\[
\varphi = \frac{\sum_{i=1}^{N}\varphi_i}{N}
\]

Случайная погрешность: 
\[
\sigma_{\text{сл}} = \sqrt{\frac{\sum_{i=1}^{4}(\varphi_{\text{ср}} - \varphi_{i})^2}{N \cdot (N - 1)}}
\]

Абсолютная погрешность угла:
\[
\sigma_{\varphi} = \sqrt{\sigma_{\text{сл}}^2 + \sigma_{\text{сист}}^2} 
\]

Итоговая ошибка для длины волны:
\[
\sigma_{\lambda} = \lambda \cdot \sqrt{\frac{\sigma_{\varphi}}{\varphi}^2 + \frac{\sigma_c}{c}^2 + \frac{\sigma_{\varphi_0}}{\varphi_0}^2 + \frac{\sigma_{\lambda_0}}{\lambda_0}^2}
\]

\begin{table}[h]
    \centering
    \begin{tabular}{|l|l|l|}
        \hline
        & $\varphi, \;^\circ$ & $\lambda, $\AA \\ \hline
        $ H_\alpha$ & $2430\pm 2$      & $6563\pm 20$        \\ \hline
        $H_\beta$   & $1445\pm 2$      & $4860\pm 20$         \\ \hline
        $H_\gamma$  & $807\pm 2$      & $4338\pm 20$         \\ \hline
        $H_\delta$  & $396\pm 2$       & $4103\pm 20$         \\ \hline
    \end{tabular}
    \caption{Спектральные линии водорода}
    \label{tab:H}
\end{table}

В пределах погрешности полученные значения совпадают с табличными для серии Бальмера. Определим постоянную Ридберга. Построим график зависимости $ \lambda^{-1} $ от выражения вида $ 1/2 - 1/m^2 $.

\begin{figure}[H]
  \centering
    \includegraphics[width=0.5\textwidth]{./images/graph_2.png}
    \caption{Зависимость $\lambda^{-1}$ от $\frac{1}{4} - \frac{1}{m^2}$}
\end{figure}

Из графика видно, что $ k = (109.8 \pm 1.1)\cdot 10^{-5} \text{\AA}^{-1}$. Тогда
\[
R = k \cdot 10^8 = 109800\pm 1100 \;\text{см}^{-1}.
\] 

Это значение совпадает с табличным в пределах погрешности.

\subsection*{Исследование спектра йода}

Теперь измерим спектр молекулы йода. Определим на монохроматоре деления, соответствующие длинноволновой линии и линии, отстоящей на 6 от нее, а также границу схождения спектра.

\begin{table}[h]
    \centering
    \begin{tabular}{|l|l|l|l|}
        \hline
        & $\varphi, \;^\circ$ & $\lambda, $\AA &  $h\nu$, эВ \\ \hline
        $1,0$ & $2291\pm 2$      & $6195\pm 20$   & $1.99 \pm 0.01$    \\ \hline
        $1,5$   & $2188\pm 2$      & $5961\pm 20$ & $2.06 \pm 0.01$      \\ \hline
        гр  & $1685\pm 2$      & $5134\pm 20$     & $2.40 \pm 0.01$   \\ \hline
    \end{tabular}
    \caption{Спектральные линии йода}
    \label{tab:H}
\end{table}

\begin{figure}[H]
  \centering
  \begin{minipage}[b]{0.45\textwidth}
    \includegraphics[width=\textwidth]{./images/spectr1.png}
  \end{minipage}
  \hfill
  \begin{minipage}[b]{0.45\textwidth}
    \includegraphics[width=\textwidth]{./images/spectr2.png}
  \end{minipage}
  \caption{Спектр йода}
\end{figure}

Вычислим в электрон-вольтах энергию колебательного кванта возбуждённого состояния молекулы йода:

\[
h\nu_2 = 14.0 \pm 0.1 \; \text{мэВ}
\]

Найдём параметры диссоциации молекул йода:
\[
h\nu_{\text{эл}} = h\nu_{1, 0} + \dfrac{3}{2}h\nu_1 - \dfrac{1}{2}h\nu_2 = 2.02\pm 0.02\; \text{эВ}
\]

Тогда для энергии диссоциации частиц в основном и возбуждённом состояниях:
\[
D_1 = h\nu_{\text{гр}} - E_A = 1.46\pm 0.01\; \text{эВ}
\]
\[
D_2 = h\nu_{\text{гр}} - h\nu_{\text{эл}} = 0.38\pm 0.02\; \text{эВ}
\]
Здесь $ E_A = 0,94\; \text{эВ}$ -- энергия возбуждения атома.

\section*{Вывод}

Таким образом, в ходе работы были получены следующие результаты:

\begin{itemize}
    \item Изучены оптические спектры водорода и йода
    \item Экспериментально проверена справедливость формулы Бальмера и посчитана постоянная Ридберга:
    \[
    R = 109800\pm 1100 \;\text{см}^{-1}.
    \]
    В пределах погрешности, она совпадает с табличной:
    \[
    R_{\text{theor}} = 109677 \;\text{см}^{-1}
    \]
    \item Оценены энергии квантов возбужденного состояния молекулы, энергия диссоциации частиц и энергия электронного перехода:
    \[
    h\nu_{\text{эл}} = 2.02\pm 0.02\; \text{эВ}
    \]
    \[
    D_1 = 1.46\pm 0.01\; \text{эВ}
    \]
    \[
    D_2 = 0.38\pm 0.02\; \text{эВ}
    \]
\end{itemize}

\end{document}